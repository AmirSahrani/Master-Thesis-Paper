% Appendix A


\chapter{Additional Material}
\label{AppendixA} % For referencing this appendix elsewhere, use \ref{AppendixA}


\section{Extended Proof}
\label{AppendixA:proof} % For referencing this appendix elsewhere, use \ref{AppendixA}

We present the following extension to the proof of \Cref{prop:rad_roy_delib},
specifically for the case where the CS distance is used.

\begin{proofc}
	As in the KS and DP cases, we construct profiles \( R_1 \), \( R_j \), and \( R_1' \) such that:
	\begin{itemize}
		\item \( \operatorname{Dist}_{\text{CS}}(R_1, R_j) = 2 \) for all \( j \neq 1 \),
		\item \( \operatorname{Dist}_{\text{CS}}(R_1, R_1') = 2 \),
		\item \( \operatorname{Dist}_{\text{CS}}(R_1', R_j) = 4 \).
	\end{itemize}

	Assume voter 1 has a bias of 1, and all other voters \( j \neq 1 \) have bias 0.5.

	Let the profiles be defined as follows:
	\[
		\begin{aligned}
			R_1  & = a \pref b \pref c \pref \cdots \pref m, \\
			R_j  & = b \pref a \pref c \pref \cdots \pref m, \\
			R_1' & = a \pref c \pref b \pref \cdots \pref m.
		\end{aligned}
	\]

	Observe that \( R_1 \) differs from both \( R_j \) and \( R_1' \) by a single adjacent transposition, and hence the CS distance between them is 2:
	\[
		\operatorname{Dist}_{\text{CS}}(R_1, R_j) = \operatorname{Dist}_{\text{CS}}(R_1, R_1') = 2.
	\]

	To compute the CS distance between \( R_1' \) and \( R_j \), consider the rankings of the top three candidates:
	\[
		\begin{aligned}
			\text{Positions in } R_1': & \quad a = 1,\; b = 3,\; c = 2, \\
			\text{Positions in } R_j:  & \quad a = 2,\; b = 1,\; c = 3.
		\end{aligned}
	\]
	Then:
	\[
		\operatorname{Dist}_{\text{CS}}(R_1', R_j)
		= |1 - 2| + |3 - 1| + |2 - 3| = 1 + 2 + 1 = 4.
	\]

	This satisfies the required conditions: the misreported preference \( R_1' \) increases the distance to other voters while remaining close to the voter’s true preference \( R_1 \), making strategic manipulation beneficial under this setup.
\end{proofc}



\newpage

\section{Additional Figures}
\label{AppendixB:figure}

\begin{figure}[ht]
	\begin{center}
		\includegraphics[width=.7\textwidth]{Figures/knowledge_pbs_dist.png}
	\end{center}
	\caption{The distribution of knowledge scores for different ranges of policy-based ideology scores.}\label{fig:knowledge_pbs}
\end{figure}

