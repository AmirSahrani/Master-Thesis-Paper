\newpage
\chapter{Discussion}
\label{Discussion}
\lhead{\emph{Discussion}}

\section{Conclusion}

The main goal of the thesis was to get a deeper understanding of deliberation
and its effect on preference profiles. To this end we consulted the literature
(\Cref{Literature}) laying out various points of view on the goal of
deliberation. From this we follow Cohen's
\cite{cohenDeliberationDemocraticLegimitimacy2002} four tenants of
deliberation, deliberation should be \textit{free, reasoned, equal}, and  it
should aim to reach \textit{consensus}. In \Cref{theory} we show that the
deliberative procedure posited by
\citet{radDeliberationSinglePeakednessCoherent2021} cannot be strategyproof
under classic notions of strategyproofness as well as novel notion of
strategyproofness we define. We use this add one more tenant to Cohen's four,
namely \textit{honesty}.

We then set out to mechanically understand deliberation, for this we introduced
the DeGroot learning model, and adapted it to deliberation over opinions. We showed
NP-hardness on the $\delta$-DBVM(S) problem, and conclude that using de DeGroot model
to model sparse graphs is computationally difficult, if one want to assign voters to nodes
based on some distances metrics.

In \Cref{experiment_results} replicated the results by
\citet{radDeliberationSinglePeakednessCoherent2021}, and we use our adapted
DeGroot model to test its predictive power on the opinion using the
\textsc{America in one Room} dataset \cite{fishkinCanDeliberationHave2024}. We
conclude that though in the first time step the model can do well on the
population level, the prediction on the change in opinion for individuals was
poor. We also show that this is at least partly explain by the fact that the
DeGroot model treats all policies equally. The data showed that some topics had
large shifts in opinions, while others showed less, the DeGroot model was
unable to capture this.

Using sensitivity analysis we showed that all parameters effected the final
predictions, but interestingly some parameters had non-significant first order
effects. We argue that this is a result of these parameters not introducing new
information, as a result they can only affect the variance of the model by
modulating the dynamics induces by the parameters with significant first order
effects.

Finally, we looked at the preference profiles which we simulated based on the
opinions from both the data and the simulations. We show that similar to the
population level predictions for the PBS, the profiles based on the simulated
and true opinions start looking more similar during the first steps in the
simulation, however, after this the model convergence too strongly and the profiles
of the simulated opinions become too ``nice''. In the sense that they get closer
to being single-peaked and are acyclic more frequently.

These results led us to conclude that the DeGroot learning model was overly
simplistic and therefore was unable to adequately explain individual opinion
change. As a result it is a bad approximation of what happens during human
deliberation. These patterns are also in contradiction to know results in
social psychology, where small extreme groups tend to become more extreme \cite{myersPolarizingEffectGroup1975}.

\section{Discussion}

We first present some limitations of these results, we can broadly put these into
3 categories.

Firstly, given the lack of a complete data source combining pre- and
post-deliberation opinions and preference rankings as well as the opinions of
these alternatives, we have had to make many assumptions on both the positions
represented by the candidates, and well as the method by which voters generate
their preferred rankings over them. In terms of generating candidates, our
approach is simple, and only assume that candidates represent the opinions held
by the voters. This is however clearly a less rich process than that by which
real world candidates are selected, where these might bring in new opinions or
have traits that are desirable, such as being good leaders or well-spoken. In
terms of voters creating a ranking over alternatives, we have gone with the
assumptions that this is done strictly through distance in opinions. Similar to
what a political compass test might do. In reality however, voters might be
using different and multiple heuristics to order the candidates, indeed if
there are numerous candidates, the ranking might not even be complete.
Therefore, distance based measure will likely diverge from heuristics, such as
pre-selecting some list of candidates deemed acceptable.


Secondly, there are some methodological assumptions we made, these mainly
relate to the generation of the trust matrices. For all Knowledge,
Self-Knowledge, and Similarity the scores were normalized to be between 0 and
1, while the Ego score was not normalized. This results in an asymmetry that
allows Ego to increase the values in the trust matrix, where the other
parameters could not. This decision was made as we found no clear ceiling Ego
with respect to which we could normalize the score. As mentioned in \Cref{experiment_results},
this might explain why Ego resulted in the lowest error on the Population level.

The same trust matrix was used for substantive and meta deliberation. Though
from a modeling perspective this is a pragmatic solution. In reality this
assumption seems too strong. This assumption forces someone to be equally
willing to change their opinion as they are to change their perception of a
candidate's opinion. Where, at least intuitively, one might expect more
willingness on the latter towards people with dissimilar opinions.


Apart from these limitations in generating the trust matrices, we also note the
noise added to the estimates of candidates opinion is normally distributed.
Though this was done to introduce voter uncertainty, over which they could then
deliberate, normally distributed noise seems unlikely, especially for voters
that hold more extreme positions. Here we might expect that the noise is
dependent on the candidates opinions, where candidates that are more similar in
opinion to the voters, will be more accurately estimated than dissimilar
candidates. For these dissimilar candidates, it might then also be true that
this noise is skewed towards the opposite extreme w.r.t. the voter's opinion.

While we opted for the DeGroot model as a more accurate representation of human
belief updating than full Bayesian updating. The DeGroot model does have some
inherent limitations. Firstly, it does not take into account why people hold
certain beliefs, nor does it constrain what kinds of beliefs a voter can hold
at the same time. To remedy this, one might consider a framework such as
abstract argumentation theory \citet{dungAcceptabilityArgumentsIts1995}, as
this is able to model the arguments with the deliberative groups. Though, this
is theoretically nice, as it allows for formal description on why opinions and
preferences are held, not just their descriptions. From a simulation
perspective, such a framework introduces major validity questions. Firstly the
framework requires a map on the relation of all arguments, for this one does
not only need qualitative data, i.e. reported arguments by participants, but
also a method of reliably and accurately transforming these qualitative reports
to argumentative graphs. Secondly, the abstract argumentation framework does
not pose an updating mechanism, thus the method through which participants
would update their believes using this framework is unclear. Secondly, it limits
voter's belief updates to linear transformations.



Finally, we address some limitations on the real world implications of these results.
The negative results surrounding strategyproofness in \Cref{theory}, might be less
of an issue in human deliberation, as the dishonest participant could be less convincing
defending their dishonest opinion than their true opinion. As a result they might have less
total influence than if they had defended their true opinion.

In terms of modeling deliberation, we have now focussed on variables that can
clearly be measured. While this might paint a good picture of the quantitative
aspects of deliberation, in practice deliberation in humans come with rich interactions
affecting their judgement and willingness to listen among other things. If we hope to
get an accurate mechanistic model of deliberation, these qualitative aspects of deliberation
need to be studied.









\section{Future work}

Based on the limitations of this study, and the literature it was based one, we present some areas for future work.

Given the weak performance of the model, a better computational model is needed
to understand deliberation and inform the design of deliberative interventions.
We propose some extensions to the model, which might better capture human
dynamics. Most importantly, it needs to be able to show non-linear affects, and
be informed by qualitative descriptions of deliberation. One main improvement of the DeGroot model specifically could be to introduce dynamic trust matrices.
When humans deliberate, the amount of trust placed on each person is likely not
fixed over time. This can be addressed dynamic trust matrices that update according
to voter's familiarity with other voters, and possibly other factors.

Another way in which the trust matrices can be further refined is through
introducing topic dependent trust. As some topics might be more hotly debated,
for example as a result of some recent event. These voters could generally be
more informed on these topics, and less willing to talk about other topics.
This is related to the notion of \textit{Salience} as described by
\citet{listDeliberationSinglePeakednessPossibility2013}, stating that topics
with high salience benefit less from deliberation, as participants have likely
received more information on this topic.

Furthermore, any good model will need proper data, as such a study similar to
that of \citet{fishkinCanDeliberationHave2024} is needed, where voters are
asked not only for their opinion but also their preference order. This could
also be a great opportunity to gather qualitative insights into deliberation
and the social dynamics there of. This would also allow for testing participant's
knowledge on topics directly, hopefully giving stronger indications of voter's ability
to persuade and defend on specific topics.

