\newpage
\chapter{Introduction}
\label{Introduction}
\lhead{\emph{Introduction}} % Set the left side page header to "Symbols"

Claims such as ``vaccines are deadly'' and ``nuclear energy is dangerous'' run
counter to expert consensus, despite experts overwhelmingly vouching for both
their safety and efficacy\footnote{While nuclear energy has seen catastrophic
	failures, such as the Chernobyl Disaster, evidence suggests that, on
	average, it results in fewer deaths and less environmental harm than
	fossil fuel-based energy \cite{ritchieWhatAreSafest2020}. This nuance
	does not contradict the broad expert consensus supporting its relative
	safety.}. Many democracies suffer this kind of misinformation, leading to a
general dissatisfaction among the electorate. Misinformation not only pushes
voters to more extreme opinions, but skews their views of fellow citizens.
Elections thus face a dual challenge: not only must they select broadly
appealing candidates, but they must do so in a context where people have
drastically differing opinions on the nature of the problems, the possible
solutions and the roles of the candidates.

For democracy to function effectively, voters need a shared foundation of
understanding --- a ``shared reality.'' This consists of commonly accepted
facts and causal relationships allowing meaningful debate about values and
priorities. For example, while nuclear energy is considered safe by experts, it
comes at high initial cost, and long construction times.
Renewable sources such as solar and wind, by contrast, can be scaled up quickly but provide less
consistent energy output. When voters share this understanding,
they can engage in productive disagreement about whether the time and money
investments for nuclear are worth the consistent energy production. However,
when some voters believe nuclear to be unsafe, an election seemingly about the
trade-off between nuclear and solar becomes a referendum on the perceived safety
of nuclear energy.


Traditionally, people's understanding of the world was shaped by family,
friends, and in legacy media. These sources tend to reinforce shared
viewpoints, friends and family often consumed similar media and held similar
beliefs, while newspapers and broadcasters curated a common public narrative
--- even if this narrative is not entirely factually accurate. Increasingly,
however, algorithmic curation shapes individual worldviews creating a
fundamental problem: a fragmented understanding of reality. Because algorithms
tailor content to each individual's preferences, people are exposed to unique
and sometimes incompatible sets of claims about the world.

This fragmentation creates a problem for collective decision-making. Voters
might be supporting the same candidate for fundamentally different, and
possibly opposing, reasons. In this work we formalize this notion of a ``reason''
using the concept of the \textit{issue dimension} introduced by
\citet{listTwoConceptsAgreement2002}, when people have a common issue
dimension, their disagreement over outcomes can be explained
through different trade-offs along these dimensions.

From the perspective of social choice, shared issue dimensions can be
beneficial. In particular if the problem can be reduced to a singular shared
issue dimension,  we might get ``single-peaked'' preferences, a special
structure in the preferences of voters.  We provide a formal definition in
\Cref{Literature}, informally however, single-peaked preferences allow for
election mechanisms that encourage voters to report their preferences honestly.
We elaborate on what we mean by an election mechanism in \Cref{chap:
	preliminaries}, but intuitively, it is a procedure for aggregating individual
preferences into a collective choice.

To promote the single-peakedness of preferences,
\citet{listDeliberationSinglePeakednessPossibility2013} propose deliberation as
a potential strategy, building on List's earlier concept of
\emph{meta-agreement} \cite{listTwoConceptsAgreement2002}, being the idea that
voters agree on which issue dimensions matter and where candidates stand
on these dimensions. \citet{listDeliberationSinglePeakednessPossibility2013} argue that deliberation can help voters
develop more coherent preference structures. Deliberation, then, helps
restructure voters' opinions in a more coherent way, particularly on
low-salience issues that receive little media coverage.

Given deliberation's potential to generate meta-agreement and more structured
preferences, we aim to understand deliberation more rigoriously.  With the rise
of in-silico experiments in computational social science, we take a
computational approach to understanding deliberation. Specifically, we adapt
the classic DeGroot model of opinion dynamics
\cite{degrootReachingConsensus1974} to the context of political deliberation,
both on voter opinions and perceived candidate positions.

To this end, we begin in \Cref{Literature} we review work on
single-peakedness, deliberation, and experiments, and present a
deliberation model by \citet{radDeliberationSinglePeakednessCoherent2021}. In
\Cref{theory} we formally define some properties of deliberation, and prove
negative results regarding ``Honesty'' during deliberation, showing
deliberation is not strategyproof under a variety of circumstances. We also
define an adaptation to the DeGroot model, as a mechanistic explanation of
deliberation through a computational model. In doing so, we find a limitation
in the applicability of  this model in the form of a negative computational
complexity result. Specifically, we show NP-completeness of mapping voter opinions to trust
matrices. In \Cref{Methods} we explain the experimental setup we use to test
our model, the result of which we present in \Cref{experiment_results}.
Finally, we reflect on the results, and broader implications of this thesis in \Cref{Discussion}.

