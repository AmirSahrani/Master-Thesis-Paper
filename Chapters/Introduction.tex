\newpage
\chapter{Introduction}
\label{Introduction}
\lhead{\emph{Introduction}} % Set the left side page header to "Symbols"

Though Black's result~\citep{blackRationaleGroupDecisionmaking1948} is a famous positive result, it is far from the only positive result relating to domain restrictions. We first look into various domain restrictions and their properties. To fully understand why single peakedness is specifically desirable we outline the political and philosophical reasons first, after which we elaborate on the mechanism through which deliberation should result in single peakedness according to \citet{listTwoConceptsAgreement2002}. Finally, for completeness' sake, we mention critiques of this theory.

% \begin{theorem}
% \end{theorem}
% \begin{theorem}
% If \(a > b\), then \(a^2 > b^2\).
% \end{theorem}
%
% \begin{corollary}
% If \(a > b > 0\), then \(a^3 > b^3\).
% \end{corollary}
%
% \begin{proposition}
% If \(a > b > 0\), then \(a^3 > b^3\).
% \end{proposition}
%
% \begin{definition}{Continuous}{continuous}
% 	{Function that...}
% \end{definition}
%
% \begin{definition}{Discrete}{disc}
% 	{Function that...}
% \end{definition}
%
%
% \begin{example}{Showing this}{}
% 	This is text
% \end{example}
%
%
% \begin{example}{A Simple Example}{simple}
% This is an example to demonstrate referencing.
% 	\hfill
% \end{example}
