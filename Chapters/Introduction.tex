\newpage
\chapter{Introduction}
\label{Introduction}
\lhead{\emph{Introduction}} % Set the left side page header to "Symbols"

Many Democracies suffer from polarization and misinformation, leading to a
general dissatisfaction with the process under voting populations. As a result
of this polarization and misinformation, voters get more extreme opinion, but
also more skewed views of the other end of the spectrum. As a result any
elections not only have the difficult task of electing candidates that please
as many people as possible, they have to do this while people have different
idea's about the problems and candidates in the first place.

For democracy to function effectively, voters need a shared foundation of
understanding, what we term a ``shared reality.'' This shared reality consists
of commonly accepted facts and causal relationships that allow meaningful
debate about policy trade-offs. For example, most economists agree that raising
minimum wages typically increases costs for businesses, which may lead to
higher consumer prices. When voters share this understanding, they can engage
in productive disagreement about whether wage increases are worth the potential
price increases. However, when some voters believe wage increases lead to
increased prices while others believe these to be unrelated, an election
seemingly about purchasing power policies becomes an inaccurate measure of
people's true intentions.

Historically, people's understanding of the world came primarily from personal
networks, family, friends, and colleagues who shared similar experiences and
information sources. Increasingly, however, algorithmic curation shapes
individual worldviews. This creates a fundamental problem: because algorithms
tailor content to individual preferences, each person may encounter a unique
set of claims about how the world works, leading to fragmented understandings
of reality.

As a result of fragmented world views, what was once a clear problem can become
messy with alternate unrelated problems. As a corollary to this, during
elections people might be voting in favor of some outcome for different, and
possibly opposing reasons. For example voting for some candidate thinking they
will lower grocery prices while also increasing the minimum wage, which, in
general are opposing forces. We later formalize this notion of a ``reason''
through the concept of the \textit{Issue Dimension}
\cite{listTwoConceptsAgreement2002}, when people have a common \textit{Issue
	Dimension}, all their differences in opinion on outcomes can be explained
through these dimensions. In the case of higher wages v.s. lower prices, for
example, a difference in opinion could be explained as favoring higher wages
as general costs will rise proportionally less.

From a social choice perspective, having shared issue dimensions can be very beneficial, especially if
in the case of a singular dimension. In this special case, we
might get ``single-peaked'' preferences, . In \Cref{Literature} we explain what
this means exactly, and why it is desirable. For now, it
suffices to note that they allow for elections mechanisms which encourage
voters to be honest about their preferences. We explain what we mean exactly by
an election mechanism in \Cref{chap: preliminaries}, but intuitively it is
simply a way to pick a winner from a set of preferences.

To increase the single-peakedness of voter preferences,
\citet{listDeliberationSinglePeakednessPossibility2013} propose deliberation
as a potential solution. Building on List's \cite{listTwoConceptsAgreement2002}
concept of meta-agreement, the idea that voters must agree on which issue
dimensions matter and where candidates stand on these dimensions. List et al. argue
that deliberation can help voters develop more coherent preference structures.
This is particularly valuable for low-salience issues that receive little
public attention, where deliberation can both increase voter knowledge and
produce preference profiles closer to single-peaked

Given deliberation's potential to lead to this meta-agreement and possibly to
nicely structured preferences, we think it important to understand
deliberation more deeply.  As in-silico experiments become more important parts
of socio-political modeling, we choose a computational approach to
understanding deliberation. Specifically, we adapt the DeGroot model
\cite{degrootReachingConsensus1974} to fit deliberation on voter opinions and
candidate's positions.

To this end, in \Cref{Literature} we present relevant work on
single-peakedness, deliberation, and experiments, as well as presenting a
deliberation model by \citet{radDeliberationSinglePeakednessCoherent2021}. In
\Cref{theory} we formally define some properties of deliberation, and prove
negative results regarding ``Honesty'' during deliberation, showing
deliberation is not strategyproof under a variety of circumstances. We also
define an adaptation to the DeGroot model, as a mechanistic explanation of
deliberation through a computational model. In doing so, we find a limitation
in the applicability of  this model in the form of a negative computational
complexity result, we show NP-completeness of mapping voter opinions to trust
matrices. In \Cref{Methods} we explain the experimental setup we use to test
our model, the result of which we present in \Cref{experiment_results}.
Finally, we reflect on the results and methods in \Cref{Discussion}.

