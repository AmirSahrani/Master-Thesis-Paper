\newpage
\chapter{Introduction}
\label{Introduction}
\lhead{\emph{Introduction}} % Set the left side page header to "Symbols"

Many Democracies suffer from polarization and misinformation, leading to a
general dissatisfaction with the process under voting populations. As a result
of this polarization and misinformation, voters get more extreme opinion, but
also more skewed views of the other end of the spectrum. As a result any
elections not only have the difficult task of electing candidates that please
as many people as possible, they have to do this while people have different
idea's about the problems and candidates in the first place.

For democracy to be effective, a shared reality is needed \cite{}.
Specifically, a shared reality is a common understanding of the world, thus
allowing people to argue from mutually agreed on premises. Historically, this
understanding of the world might have come from people such as family, friends
and colleagues. Increasingly, however, a person's sense of reality is defined
through algorithmic means. This raises a problem, as on many social media
networks, algorithms are tailored to individuals' preferences, and as a result,
each individual might end up seeing a unique set of propositions about the
world.

As a result of fragmented world views, what was once a clear problem can become
messy with alternate unrelated problems. As a corollary to this, during
elections people might be voting in favor of some outcome for different, and
possibly opposing, reasons. For example voting thinking some candidate is going
to lower grocery prices while also increasing the minimum wage, which, in
general are opposing forces. We later formalize this notion of a ``reason''
through the concept of the \textit{Issue Dimension}
\cite{listTwoConceptsAgreement2002}, when people have a common \textit{Issue
	Dimension}, all their differences in opinion on outcomes can be explained
through these dimensions.

From a social choice perspective, having shared issue dimensions, especially if
it is a singular dimension, can be very beneficial. In this special case, we
might get ``single-peaked'' preferences. In \Cref{Literature} we explain what
these types of preferences are, and why they are desirable. For now, it
suffices to note that they allow for elections mechanisms which encourage
voters to be honest about their preferences. We explain what we mean exactly by
an election mechanism in \Cref{chap: preliminaries}, but intuitively it is
simply a way to take all the preferences (as reported through a ballot) and
pick a winner from them.

\citet{listTwoConceptsAgreement2002} pose deliberation as a method to increase
the ``single-peakedness'' of the preferences of participants. For this List ...
introduce the notion of meta-agreement, which captures the idea of agreeing on
which issue dimensions are relevant, as well as where the candidates fall
within these issue dimensions. Furthermore, deliberation has been shown to
reduce polarization, .... , [SOURCES].

Given this we think it important to understand deliberation more deeply. To
this end, in \Cref{Literature} we present relevant work on single-peakedness,
deliberation, and experiments. In \Cref{theory} we formally define some
properties of deliberation, and prove generally negative results regarding
``Honesty'' during deliberation. We also define an adaptation to the DeGroot
model, as a mechanistic explanation of deliberation, as well as a negative
result for using this model outside deliberative settings. In \Cref{Methods}
w..e explain the experimental setup we use to test our model, the result of
which can be found in \Cref{experiment_results}. Finally, we reflect on the
results and methods in \Cref{Discussion}.

