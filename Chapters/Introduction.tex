\newpage
\chapter{Introduction}
\label{Introduction}

\begin{theorem}
	This theorem must look nice!
\end{theorem}
\begin{theorem}
If \(a > b\), then \(a^2 > b^2\).
\end{theorem}

\begin{corollary}
If \(a > b > 0\), then \(a^3 > b^3\).
\end{corollary}

\begin{proposition}
If \(a > b > 0\), then \(a^3 > b^3\).
\end{proposition}

\begin{definition}
A function that...
\label{def:Continuous}
\end{definition}

\begin{definition}
A function that...
\label{def:Discrete}
\end{definition}


\begin{example}{Showing this}{}
	This is text
\end{example}


\begin{example}{A Simple Example}{simple}
This is an example to demonstrate referencing.
	\hfill
\end{example}

In \cref{ex:simple} we can see what is happening. 

\begin{lstlisting}[language=caml]
let agent = 1 in
let text = "Testing" in
(* This is a comment *)
print_int agent
\end{lstlisting}

