\newpage
\chapter{Introduction}
\label{Introduction}
\lhead{\emph{Introduction}} % Set the left side page header to "Symbols"


In a polarized world (yuck), not only do people's opinions get more opposed,
their view of others also becomes less accurate.

For democracy to be effective, a shared reality is needed [SOURCE].
Specifically, a shared reality is a common understanding of the world, thus
allowing people to argue from mutually agreed on premises. Historically, this
understanding of the world might have come from people such as family, friends
and colleagues. Increasingly, however, a person's sense of reality is defined
through algorithmic means. This raises a problem, as on many social media
networks, algorithms are tailored to individuals' preferences, and as a result,
each individual might end up seeing a unique set of propositions about the
world.

As a result of fragmented world views, what was once a clear problem can become
messy with alternate unrelated problems. For example, ...

As a corollary to this, during elections people might be voting in favor of
some outcome for different, and possibly opposing, reasons. We later formalize
this notion of a ``reason'' through the concept of the \textit{Issue Dimension}
\cite{listTwoConceptsAgreement2002}, when people have a
common \textit{Issue Dimension}, all their differences in opinion on outcomes
can be explained through these dimensions.

From a social choice perspective, having shared issue dimensions, especially if
it is a singular dimension, can be very beneficial. In this special case, we
might get ``single-peaked'' preferences. In \Cref{Literature} we explain what
these types of preferences are, and why they are desirable. For now, it
suffices to note that they allow for elections mechanisms which encourage
voters to be honest about their preferences. We explain what we mean exactly by
an election mechanism in \Cref{chap: preliminaries}, but intuitively it is
simply a way to take all the preferences (as reported through a ballot) and
pick a winner from them.

% Link this to epistemic trust, as well as deliberative interventions with positive effects.
\citet{listTwoConceptsAgreement2002} pose deliberation as a method to increase
the ``single-peakedness'' of the preferences of participants. For this List ...
introduct the notion of meta-agreement, which captures the idea of agreeing on
which issue dimensions are relevant. Furthermore, deliberation has been shown
to reduce polarization, .... , [SOURCES].

Given this we think it important to understand deliberation more deeply. To
this end, in \Cref{Literature} we present relevant work on single-peakedness,
deliberation, and experiments. In \Cref{theory} we formally define some
properties of deliberation, and prove generally negative results regarding
``Honesty'' during deliberation. We also define an adaptation to the DeGroot
model, as a mechanistic explanation of deliberation, as well as a negative
result for using this model outside deliberative settings. In \Cref{Methods}
w..e explain the experimental setup we use to test our model, the result of
which can be found in \Cref{experiment_results}. Finally, we reflect on the
results and methods in \Cref{Discussion}.


%  Take for example a vote on dinner plans. Alice, bob and charlie all
% agree to eat pizza, but need to decide on between 3 restaurants, (a) ... (b)
% ... (c) ..., Alice and Bob agree that restaurant (a) is the cheapest, while
% serving the least tasty pizza, (c) is the most expensive, but has the best
% pizza in the city. Finally, (b) is a compromise between the two. Given that Bob
% is a little short on money, he would like the cheapest restaurant the group can
% agree on, while Alice really wants the best possible pizza. For Charlie, we
% consider two cases, the ``shared'' case, in which he sees the same trade-offs,
% and the ``fragmented'' case, where he has his own issue dimension. If Charlie
% shared the issue dimension, he does not mind spending a little bit extra for
% better pizza, really does not want to go to the most expensive restaurant.
% Given this, it is reasonable to settle on (b), we make the notion on why this
% is clear in \Cref{Literature}. Now if Charlie instead things that restaurant
% (c) is most likely to serve his favorite drink, but somehow is oblivious on the cost of pizza 
% or its quality. As a result restaurant (c) now has 




% Though Black's result~\citep{blackRationaleGroupDecisionmaking1948} is a
% famous positive result, it is far from the only positive result relating to
% domain restrictions. We first look into various domain restrictions and their
% properties. To fully understand why single-peakedness is specifically
% desirable we outline the political and philosophical reasons first, after
% which we elaborate on the mechanism through which deliberation should result
% in single-peakedness according to \citet{listTwoConceptsAgreement2002}.
% Finally, for completeness' sake, we mention critiques of this theory.
