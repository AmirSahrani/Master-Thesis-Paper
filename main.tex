%%%%%%%%%%%%%%%%%%%%%%%%%%%%%%%%%%%%%%%%%
% Masters/Doctoral Thesis 
% LaTeX Template
% Version 1.43 (17/5/14)
%
% This template has been downloaded from:
% http://www.LaTeXTemplates.com
% Original authors:
% Steven Gunn 
% http://users.ecs.soton.ac.uk/srg/softwaretools/document/templates/
% and
% Sunil Patel
% http://www.sunilpatel.co.uk/thesis-template/
%
% License:
% CC BY-NC-SA 3.0 (http://creativecommons.org/licenses/by-nc-sa/3.0/)
%
% Note:
% Make sure to edit document variables in the Thesis.cls file
%
%
% Modified and Adapted by Michael Lees 2020
%%%%%%%%%%%%%%%%%%%%%%%%%%%%%%%%%%%%%%%%%

%----------------------------------------------------------------------------------------
%    PACKAGES AND OTHER DOCUMENT CONFIGURATIONS
%----------------------------------------------------------------------------------------

\documentclass[11pt, oneside, dvipsnames]{Thesis} % The default font size and one-sided printing (no margin offsets)
\graphicspath{{Figures}} % Specifies the directory where pictures are stored


\usepackage[square, numbers, comma, sort]{natbib} % Use the natbib reference package - read up on this to edit the reference style; if you want text (e.g. Smith et al., 2012) for the in-text references (instead of numbers), remove 'numbers' 
\usepackage[ruled,vlined]{algorithm2e}
\usepackage{amsmath}
% \usepackage{subfig}
\usepackage{subcaption}

\usepackage{graphicx}
\usepackage{booktabs}


\usepackage{enumitem}
\setlist{nolistsep}
\usepackage{booktabs}
\usepackage{algorithm2e}
\usepackage{custom_commands}

\usepackage{titlesec}
\DeclareMathOperator*{\argmin}{arg\,min}
\DeclareMathOperator*{\argmax}{arg\,max}
\title{\ttitle} % Defines the thesis title - don't touch this

\begin{document}

\frontmatter % Use roman page numbering style (i, ii, iii, iv...) for the pre-content pages
\setstretch{1.3} % Line spacing of 1.3

% Define the page headers using the FancyHdr package and set up for one-sided printing
\fancyhead{} % Clears all page headers and footers
\rhead{\thepage} % Sets the right side header to show the page number
\lhead{} % Clears the left side page header

\pagestyle{fancy} % Finally, use the "fancy" page style to implement the FancyHdr headers

\newcommand{\HRule}{\rule{\linewidth}{0.5mm}} % New command to make the lines in the title page


% PDF meta-data
\hypersetup{pdftitle=\ttitle}
\hypersetup{pdfsubject=\subjectname}
\hypersetup{pdfauthor=\authornames}
\hypersetup{pdfkeywords=\keywordnames}

%----------------------------------------------------------------------------------------
%    TITLE PAGE
%----------------------------------------------------------------------------------------

\begingroup
\hypersetup{
	urlcolor=black
}
\begin{titlepage}
	\begin{center}

		\textsc{\LARGE \univname}\\[1.5cm] % University name
		\textsc{\Large Master Thesis}\\[0.5cm] % Thesis type

		\HRule \\[0.4cm] % Horizontal line
		{\huge \bfseries \ttitle}\\[0.4cm] % Thesis title
		\HRule \\[1.5cm] % Horizontal line

		\begin{minipage}{0.4\textwidth}
			\begin{flushleft} \large
				\emph{Author:}\\
				\href{http://amirsahrani.com}{\authornames} % Author name - remove the \href bracket to remove the link
			\end{flushleft}
		\end{minipage}
		\begin{minipage}{0.4\textwidth}
			\begin{flushright} \large
				\emph{Examiner:} \\
				{\exname}\\
				\emph{Supervisor:} \\
				{\supname}\\
				\emph{Assessor:} \\
				{\assessorname}
			\end{flushright}
		\end{minipage}\\[1cm]

		\large \textit{A thesis submitted in partial fulfillment of the requirements\\ for the degree of \degreename}\\[0.3cm] % University requirement text
		\textit{in the}\\[0.4cm]
		\groupname\\\deptname\\[2cm] % Research group name and department name

		{\large \today}\\[2cm] % Date
		% \includegraphics[width=0.6\textwidth]{clslogo.png} % Include Computational Science Logo

		\vfill
	\end{center}

\end{titlepage}
\endgroup

%----------------------------------------------------------------------------------------
%    DECLARATION PAGE
%    Your institution may give you a different text to place here
%----------------------------------------------------------------------------------------

\Declaration{

	\addtocontents{toc}{\vspace{1em}} % Add a gap in the Contents, for aesthetics

	I, \authornames, declare that this thesis, entitled `\ttitle' and the work presented in it are my own. I confirm that:

	\begin{itemize}
		\item[\tiny{$\square$}] This work was done wholly or mainly while in candidature for a research degree at the University of Amsterdam.
		\item[\tiny{$\square$}] Where any part of this thesis has previously been submitted for a degree or any other qualification at this University or any other institution, this has been clearly stated.
		\item[\tiny{$\square$}] Where I have consulted the published work of others, this is always clearly attributed.
		\item[\tiny{$\square$}] Where I have quoted from the work of others, the source is always given. With the exception of such quotations, this thesis is entirely my own work.
		\item[\tiny{$\square$}] I have acknowledged all main sources of help.
		\item[\tiny{$\square$}] Where the thesis is based on work done by myself jointly with others, I have made clear exactly what was done by others and what I have contributed myself.
	\end{itemize}


	Signed:

	\vspace{1em}


	\includegraphics[width=3cm]{Figures/yoursignature.pdf} \par

	Date: \today
}

\clearpage % Start a new page

%----------------------------------------------------------------------------------------
%    QUOTATION PAGE
%----------------------------------------------------------------------------------------

\pagestyle{empty} % No headers or footers for the following pages

\null\vfill % Add some space to move the quote down the page a bit

\textit{``The majority, standing in for the people, wills everything and therefore wills nothing''}

\begin{flushright}
	Joshua Cohen
\end{flushright}

\vfill\vfill\vfill\vfill\vfill\vfill\null % Add some space at the bottom to position the quote just right

\clearpage % Start a new page

%----------------------------------------------------------------------------------------
%    ABSTRACT PAGE
%----------------------------------------------------------------------------------------

\addtotoc{Abstract} % Add the "Abstract" page entry to the Contents

\abstract{\addtocontents{toc}{\vspace{1em}} % Add a gap in the Contents, for aesthetics

	Deliberation is often proposed as a remedy to democratic dysfunction, enabling
	voters to reach more informed and coherent preferences. In particular,
	deliberation may promote a shared understanding of the relevant
	issue dimensions (meta-agreement), which can lead to single-peaked preference profiles and
	circumvent classic impossibility results in social choice theory. This thesis
	investigates whether and how deliberation fosters such structured preferences
	by adapting the DeGroot model of opinion dynamics.

	We begin by reviewing theoretical foundations from social choice,
	focusing on domain restrictions like single-peakedness, and by
	discussing deliberative democracy and the concept of meta-agreement. We
	then replicate some of the results of Rad and Roy, published in the
	\textit{American Political Science Review} in 2021, which models
	deliberation as preference updating under various distance metrics.
	While their model can increase proximity to single-peakedness, it does
	not capture meta-agreement and is vulnerable to strategic manipulation.

	To address the lack of meta-agreement, we introduce an extended
	DeGroot-based model in which agents deliberate not only over their
	substantive preferences but also over their beliefs about candidate
	positions on multiple policy dimensions. Trust dynamics are modeled via
	bias, ego, similarity, and knowledge, and are used to guide how individuals
	weight others’ opinions. Using data from the well-known \textsc{America in One
		Room} experiment, we calibrate
	and validate the model, showing it reproduces some empirical patterns
	of opinion change. Notably, we find that ego-based trust better fits
	observed data than knowledge-based trust, suggesting that being
	informed does not necessarily translate to persuasive influence.

	We further extend the model to simulate deliberation-driven
	meta-agreement and show that it can reduce cyclic preferences and
	increase proximity to single-peakedness. Finally, a sensitivity
	analysis identifies drivers of opinion change and highlights the
	interaction effects between trust, bias, and group composition. These
	results suggest that while deliberation can indeed structure
	preferences, the dynamics are more complex and depend on voters'
	understanding of the broader issue landscape. }

\clearpage % Start a new page

%----------------------------------------------------------------------------------------
%    ACKNOWLEDGEMENTS
%----------------------------------------------------------------------------------------

\setstretch{1.3} % Reset the line-spacing to 1.3 for body text (if it has changed)

\acknowledgements{\addtocontents{toc}{\vspace{1em}} % Add a gap in the Contents, for aesthetics

	I would like to thank Ulle Endriss for helping me walk the tight rope
	between scientific disciplines, a path I find myself walking all too
	often. His guidance has been of tremendous help whenever I found myself
	lost, confused or when I was simply getting ahead of myself.

	I thank my friends, and family for keeping me sane. Though stress and
	tension sometimes ran high; the long talks, cards, and plastic
	rock have kept my mind from fraying.

	Thanks, but I'll take some more fish.


}
\clearpage % Start a new page

%----------------------------------------------------------------------------------------
%    LIST OF CONTENTS/FIGURES/TABLES PAGES
%----------------------------------------------------------------------------------------

\pagestyle{fancy} % The page style headers have been "empty" all this time, now use the "fancy" headers as defined before to bring them back

\lhead{\emph{Contents}} % Set the left side page header to "Contents"
\tableofcontents % Write out the Table of Contents


\lhead{\emph{List of Figures}} % Set the left side page header to "List of Figures"
\listoffigures % Write out the List of Figures

\lhead{\emph{List of Tables}} % Set the left side page header to "List of Tables"
\listoftables % Write out the List of Tables

% \lhead{\emph{List of Algorithms}} % Set the left side page header to "List of Algorithms"
% \addtotoc{List of Algorithms}
% \listofalgorithms % Write out the List of Tables

%----------------------------------------------------------------------------------------
%    ABBREVIATIONS
%----------------------------------------------------------------------------------------

\clearpage % Start a new page

\setstretch{1.3} % Set the line spacing to 1.5, this makes the following tables easier to read

\lhead{\emph{Abbreviations}} % Set the left side page header to "Abbreviations"
\listofsymbols{ll} % Include a list of Abbreviations (a table of two columns)
{
	\textbf{PBS} & \textbf{P}olicy-\textbf{B}ased (ideology) \textbf{S}core\\
	\textbf{PtS} & \textbf{P}roximity \textbf{t}o \textbf{S}ingle-peakedness\\
	\textbf{PtS-V} & \textbf{P}roximity \textbf{t}o \textbf{S}ingle-peakedness (through) \textbf{V}oter (deletion)\\
	\textbf{PtS-C} & \textbf{P}roximity \textbf{t}o \textbf{S}ingle-peakedness (through) \textbf{C}andidate (deletion)\\
}

%----------------------------------------------------------------------------------------
%    PHYSICAL CONSTANTS/OTHER DEFINITIONS
%----------------------------------------------------------------------------------------

%\lhead{\emph{Physical Constants}} % Set the left side page header to "Physical Constants"

%\listofconstants{lrcl} % Include a list of Physical Constants (a four column table)
%{
%Speed of Light & $c$ & $=$ & $2.997\ 924\ 58\times10^{8}\ \mbox{ms}^{-\mbox{s}}$ (exact)\\
%% Constant Name & Symbol & = & Constant Value (with units) \\
%}

%----------------------------------------------------------------------------------------
%    SYMBOLS
%----------------------------------------------------------------------------------------

\clearpage % Start a new page

\lhead{\emph{Symbols}} % Set the left side page header to "Symbols"

\listofnomenclature{ll} % Include a list of Symbols (a three column table)
{
	$N$ & The set of all voters\\
	$X$ & The set of all alternatives\\
	$\pref$ & A preference relationship\\
	$\domain{ }$ & A domain of possible profiles\\
	$D$ & A deterministic deliberative procedure \\
	$\biasedDeliberation$ &  A deliberative procedure with biased voters\\
	$\setOfStrictProfiles$ & Set of all possible preference orders over A\\
	$\Preference$ & Set of a preference relations over all candidates\\
	$\strictProfile$ & Set of preferences of all voters\\
	$\votingrule{ }$ & A function mapping a strict profile to a candidate \\
	$\orderalt$ & A geometric order over candidates\\
	$\policies$ & Vector of all policies\\
	$\policy$ & An instance of a policy\\
	$\Support$ & Vector of support for each policy\\
	$\EstSupport$ & matrix of shape $|\alternatives| \times |\policies|$, estimating support of policies for each alternative\\

}

%----------------------------------------------------------------------------------------
%    DEDICATION
%----------------------------------------------------------------------------------------

\setstretch{1.3} % Return the line spacing back to 1.3

\pagestyle{empty} % Page style needs to be empty for this page
{

	\addtocontents{toc}{\vspace{2em}} % Add a gap in the Contents, for aesthetics
}

%----------------------------------------------------------------------------------------
%    THESIS CONTENT - CHAPTERS
%----------------------------------------------------------------------------------------

\mainmatter % Begin numeric (1,2,3...) page numbering

\pagestyle{fancy} % Return the page headers back to the "fancy" style

% Include the chapters of the thesis as separate  tex files from the Chapters folder
% Change the file names if you prefer

%This structure provides a bare bones essentials of the thesis and some indicative length
%This structure may not fit your thesis perfectly, but be sure to include these components somehow.
%It is possible to split the chapters up (E.g., 2 methods chapter, Experiment and Results as two, etc.)

% The typical length may be between 43 - 64 pages. Do not worry if you go slightly larger or smaller than this. 
% But a thesis of 20 pages, or 100 pages may suggest you've been to brief or too verobse.
%NOTE - there is not strict limit/minimum.

\newpage
\chapter{Introduction}
\label{Introduction}

\begin{theorem}
	This theorem must look nice!
\end{theorem}
\begin{theorem}
If \(a > b\), then \(a^2 > b^2\).
\end{theorem}

\begin{corollary}
If \(a > b > 0\), then \(a^3 > b^3\).
\end{corollary}

\begin{proposition}
If \(a > b > 0\), then \(a^3 > b^3\).
\end{proposition}

\begin{definition}[Continuous Function]
A function that...
\end{definition}


\begin{example}
	
	This is text 

\end{example}
\begin{example}[Special Case]
This is the body of the example. It can include regular text, equations, or even environments like \texttt{minipage}:
\begin{minipage}{0.5\textwidth}
Content inside a minipage.
\end{minipage}
\end{example}



\begin{lstlisting}[language=caml]
let agent = 1 in
let text = "Testing" in
(* This is a comment *)
print_int agent
\end{lstlisting}

 %Set out your thesis and state your research question (5-8 pages)
\chapter{Preliminaries}
\label{chap: preliminaries}
\lhead{\emph{Preliminaries}}

We begin with a short introduction to social choice. We outline the basic
voting model, closely following the notation and definitions by
\citet{brandtHandbookComputationalSocial2016}, and restate well-known results
relevant to the following chapters.

\section{The Basic Model}

To model elections, we represent voters by the set $\voters$ consisting of $n$
voters. The possible outcomes of an election, we represent with the set
$\alternatives$ consisting of $|A|$ possible outcomes, usually called the alternatives. In line with the topic of political elections, we will
refer to the outcomes of an election as candidates instead. Each voter;represents
their preference on candidates through a preference relation $\pref_i$, for
example if voter i prefers outcome $a$ to outcome $b$, we write $a \pref_i b$.
When a voter's preference is antisymmetric, complete and transitive, i.e. it orders
all candidates and $a \pref_i b$ and $b \pref_i c$ implies $a \pref_i c$, we
call this a linear order, denoted by $\Preference_i$. We call the set of possible linear orders over the
candidates $\setOfStrictProfiles$.  For an election, all voters report a
linear order. The vector consisting of each voter's preference is called a
profile, denoted by $\strictProfile = (\Preference_1, \dots \Preference_{n}) \in \setOfStrictProfiles^{n}$. Finally, a social choice function (SCF)
$f$ decides the outcome of the election based on the profile. We discuss the
specifics of these functions in \Cref{sec:SCF}.

The last simple definition we will need is the \emph{majority relation}
\cite{alma990028050780205131}. Given some profile $\strictProfile$ we can
construct a majority relationship as follows: for each pair of candidates
$x,y$, we ask how many voters strictly prefer $x$ to $y$; if this number of people is
greater than $\frac{n}{2}$ we get $x  \prefmaj y$. If it is exactly equal to
$\frac{n}{2}$ and thus is a tie, we simply write $x \prefeqmaj y$ (breaking tie arbitrarily), otherwise we write $y \prefmaj x$. We proceed with an
example.

\begin{example}{Majority relation}{maj-rel}
	\begin{minipage}{0.15\linewidth}
		\begin{tabular}{ccc}
			\toprule
			$1$ & $2$ & $3$ \\
			\midrule
			$a$ & $b$ & $a$ \\
			$b$ & $c$ & $c$ \\
			$c$ & $a$ & $b$ \\
			\bottomrule
		\end{tabular}
	\end{minipage}
	\hspace{0.02\linewidth}
	\begin{minipage}{0.78\linewidth}
		Given the profile on the left, we
		first start by comparing $a$ to $b$, both voters 1 and 3 prefer
		$a$ to $b$ thus the majority has prefers $a$ to $b$. Comparing
		$b$ to $c$ the majority prefers $b$ to $c$. Finally, comparing
		$a$ to $c$, $a$ is preferred again. Thus, the majority relation
		is $a \prefmaj b \prefmaj c$.
	\end{minipage}
\end{example}

From this it is easy to see that the majority relation is in some sense a
summary of the voter's preferences. In \Cref{preliminaries: negative results}
we show how a divided population can lead to an inconsistent majority relation.

Using this majority relationship, we can formulate our first notion of when an
candidate is winning. We call an candidate $x$ a Condorcet winner, if for
each pairwise comparison between $x$ and $y \in \alternatives\setminus
	\{x\}$ we have $x \prefmaj y$ for example, in \Cref{ex:maj-rel} $a$ would be
the Condorcet winner. We can relax the requirement of always winning to never
losing, i.e. we never have $y \prefmaj x$ but $x \prefeqmaj y$ is allowed. A
candidate that never loses in any pair wise comparison is called a weak
Condorcet winner.

\section{Social Choice Functions} \label{sec:SCF}

As mentioned, in order to decide the outcome of an election we need a social
choice function $f$, this function should map all possible profiles to an
outcome, thus $f: \setOfStrictProfiles^n \to \alternatives$. A famous and
simple example of a SCF is the plurality rule, which simply elects the
candidate voted into first place most often, i.e. ``most first place votes
wins''. This rule presents on of the first challenges for many SCF, it must deal with ties.

For elections organizers likely will want to ensure the SCF has certain
nice properties, such as not favoring a candidate. In social choice these
properties are called axioms, and the procedure of designing a SCF based on
desired axioms is called the axiomatic approach. The name of the property just
described is the axiom of neutrality, stating that the SCF should be
neutral with respect to the candidates. In this work six main axioms are of
importance.

\emph{Axiom of Resoluteness.} A SCF $f$ is resolute, if for every profile
$\strictProfile$ we have $|f(\strictProfile)| = 1$.

\emph{Axiom of Surjectivity.} A SCF $f$ is surjective, if for every candidate
$x$, there exists a profile $\strictProfile$ such that
$f(\strictProfile) = x$.

\emph{Axiom of Non-Dictatorship.} A SCF $f$ is non-dictatorial, if there does not exist a voter $i$ such that $f(\strictProfile) = \textnormal{top}(i,\strictProfile)$ for all profiles $\strictProfile$, where $\textnormal{top}(i,\strictProfile)$  extracts voter $i$'s most preferred candidate from profile $\strictProfile$.

\emph{Axiom of Strategyproofness.} A SCF $f$ is strategyproof if, for any
voter $i \in \voters$, $i$ cannot report an untruthful preference
$\pref_i'$, such that  $\strictProfile' = (\pref_{1}, \dots,
	\pref_{i}', \dots, \pref_{n})$ and $f(\strictProfile') \pref_i
	f(\strictProfile)$.

\emph{Axiom of Anonymity.} A SCF $f$ is anonymous if, when the labels of voters
are shuffled, the winning candidate stays the same.

\emph{Axiom of Neutrality.} A SCF $f$ is neutral if, when the labels of the
candidates are shuffled, the winning candidate in the shuffled election, is
the candidate that has the ranks of the winning candidate in the original
election.

There are many more axioms on could reasonably argue for however, these are
enough to lead to the main impossibility results this work focuses on.

\section{Negative Results}
\label{preliminaries: negative results}

Classic social choice theory has many negative results one such example is the
Condorcet cycle. This is a specific profile that results in a cycle in the
majority relation, as shown in the following example.

\begin{example}{Condorcet cycle}{condorcet-cycle}
	\begin{minipage}{0.15\linewidth}
		\begin{tabular}{ccc}
			\toprule
			$1$ & $2$ & $3$  \\
			\midrule
			$a$ & $b$ & $c $ \\
			$b$ & $c$ & $a $ \\
			$c$ & $a$ & $b $ \\
			\bottomrule
		\end{tabular}
	\end{minipage}
	\hspace{0.02\linewidth}
	\begin{minipage}{0.78\linewidth}
		Voters 1 and 3  prefer $a$ to $b$ resulting in $a \prefmaj b$,
		next voters 1 and 2 prefer $b$ to $c$, resulting in $b \prefmaj
			c$. However, voters 2 and 3 prefer $c$ to $a$, resulting in $c
			\prefmaj a$. This yields the cycle $a \prefmaj b \prefmaj c
			\prefmaj a$.
	\end{minipage}
\end{example}

It can be shown that under weak preferences the Condorcet
cycle can occur anytime there are 3 or more candidates and voters. While
under strict preferences this can occur anytime there is an odd number of preferences at least 3, with the number of voters being a multiple of the
number of candidates. As we will show later, this profile can be the cause of
some impossibility results.

One of the major negative results in social choice is that of the
Gibbard-Satterthwaite theorem
\citep{gibbardManipulationVotingSchemes1973,satterthwaiteStrategyproofnessArrowsConditions1975}.

\begin{theorem}[Gibbard-Satterthwaite]
	\label{thm:gs-thm}
	There exists no resolute social choice function for elections with $|\alternatives| \geq$ 3 that is surjective, strategyproof, and non-dictatorial.
\end{theorem}

Unless we accept a dictatorship, it is impossible to have a voting
rule that incentivizes voters to report their preferences truthfully, when we
want to pick a singular winner from at least 3 candidates.

Though we do not provide a full proof, the Condorcet cycle offers some
intuition for why this result holds. Following \Cref{ex:condorcet-cycle},
suppose we have a social choice function (SCF) $f$ that elects candidate $a$.
Voter 1 is very happy with this outcome, but voters 2 and 3 would prefer $c$
instead. Voter 2 could then misreport their preferences by swapping $c$ and
$b$, thereby causing $c$ to become the Condorcet winner.

Now, if $f$ is both strategyproof and resolute, it must still elect $a$ despite
$c$ being the Condorcet winner. Since $f$ is also surjective, $a$ cannot be the
outcome for all preference profiles. Taken together, the only apparent reason
$a$ continues to win in this profile is because voter 1 wants it to—suggesting
that voter 1 effectively dictates the outcome.

Fortunately, there seem to be ways around these negative results. Mainly
through the assumption that there is some structure in the preferences of
voters.

\section{Domain Restrictions} \label{sec:Domain-res}

Negative results often are a result of a small set of ill-behaved profiles. If there is
reason to conclude these profiles are impossible in the election at hand, there
is some hope of constructing SCF's satisfying our axioms. To speak more
formally about profiles ``not occuring'', we introduce Domain restrictions, for
this we use the definition by
\citet{elkindPreferenceRestrictionsComputational2022}.

\begin{definition}{Domain}{domain}
	{
		Given a set of voters $\voters$, candidates $A$, and conditions $C$, the domain $\domain{}$ of an election is the set of all profiles $\strictProfile$ such that all conditions $C$ are satisfied.
	}
\end{definition}

This definition is different from usual definitions in social choice in so far as it talks about allowed profiles instead of allowed votes.

As stated earlier, the Condorcet profile is one such ill-behaved profile, as
each candidate, holds a majority preference over another candidate.
Naturally one might consider if this profile might even come up in practice,
though conceivable, it seems generally unlikely for there to exist a perfect
split in opinions. Quite naturally one of the first ``solutions'' one might
consider is when the number of voters is not a multiple of the number of
candidates, though this is hardly a useful solution since it only prevents
Condorcet cycles, it is the first example of a domain restriction, we define a
simple domain that prevents these cycles as follows.

\begin{definition}{$\domain{No-tie}$}{dom-ties}
	Let $\alternatives$ be the set of candidates and $\voters$ be the set of voters, of size $n$ such that $n \neq k \cdot |\alternatives|$ for any $k \in \Nat$. We call this domain $\domain{No-tie}$.
\end{definition}

This allows us to state our first proposition.

\begin{proposition}
	The plurality rule never returns an $|\alternatives|$-way tie between candidates when applied to $\domain{No-tie}$.
\end{proposition}

\begin{proofc}
	Assume, for the sake of contradiction, the plurality rule in fact does
	return an $|\alternatives|$-way tie, this means all candidates were ranked first an
	equal number of times call this $k$. Necessarily then, we need
	exactly $k \cdot |\alternatives|$ voters, but this leads to a
	contradiction, as this would no longer be inside $\domain{No-tie}$.
\end{proofc}

This is a simple result, but it serves as an example on how we can use the
properties of the domain to prove things about the election.
\citet{gaertnerDomainRestrictions2002} establishes two ways in which a domain can
be restricted. Firstly we can restrict the domain to a number of voters or
candidates, which is what we did in $\domain{No-tie}$. Secondly, the domain can
be restricted to have a certain structure, such as being single-peaked.


In an election the candidates might represent an axis, such that a voters
prefers an candidate more if they are closer to them on the axis. For
example, if the candidates represent the minimum wage, where each cent-value
constitutes an candidate. Imagine a voter thinks the minimum wage should  be
some value $x$ and prefers candidates that are closer to this value $x$. This
results in each voter having a ``peak'' value, and all other values are ranked
in terms of their distance to $x$. \Cref{fig:singlepeaked_vis} shows what this
might look like for 3 voters. More generally, we call a profile single-peaked
if there exists an axis on which we can place the candidates such that all
voters' preferences have a single peak on this axis. \Cref{def:single-peaked}
makes this notion formal.

\begin{figure}[ht]
	\centering
	\begin{subfigure}[b]{0.3\textwidth}
		\centering
		\begin{subfigure}[b]{0.3\textwidth}
			\centering
			\begin{tabular}{ccc}
				\toprule
				$1$ & $2$ & $3$ \\
				\midrule
				$c$ & $d$ & $b$ \\
				$d$ & $c$ & $c$ \\
				$e$ & $b$ & $d$ \\
				$b$ & $a$ & $a$ \\
				$a$ & $e$ & $e$ \\
				\bottomrule
			\end{tabular}
			\vspace{2.8em}
		\end{subfigure}
		\caption{Preference profile}\label{tab:corresponding_profile}
	\end{subfigure}
	\hfill
	\begin{subfigure}[b]{0.65\textwidth}
		\centering
		\includegraphics[width=\textwidth]{Figures/single_peak_vis.png}
		\caption{Single-peaked profile visualization}\label{fig:singlepeaked_vis}
	\end{subfigure}
	\caption{An election with three voters and five candidates. Each voter has a unique peak, and the profile is single-peaked with respect to a shared axis.}
	\label{fig:singlepeaked_full}
\end{figure}
\begin{definition}{Single-peaked Profiles}{single-peaked}
	A profile $\strictProfile$ is single-peaked, if given some ordering
	$\orderalt$ over the candidates, it holds that for all voters $i$, and
	all $a, b, c \in \alternatives$, if $a \orderalt b \orderalt c$, then
	at most $a \pref_i b$ or $c \pref_i b$, but never both.
\end{definition}

This thesis will now focus on measures to ``increase'' single-peakedness of profile.



 %Set out your thesis and state your research question (5-8 pages)
\newpage
\chapter{Literature review}
\label{Literature}

\lhead{\emph{Literature Review}} % Set the left side page header to "Symbols"

Though Black's result~\citep{Black_1948} is a famous positive result, it is far from the only positive result relating to domain restrictions. We first look into various domain restrictions and their properties. To fully understand why single peakedness is specifically desirable we outline the political and philosophical reasons first, after which we elaborate on the mechanism through which deliberation should result in single peakedness according to \citet{List_2002}. Finally for completeness sake, we mention critiques of this theory.

\section{Domain Restrictions}
A voting domain $\mathcal{D}$ is the domain of all possible voting profiles \(R\) given some number of voters $N$ and some number of alternatives $|X|$. Intuitively, this is simple the space of all possible outcomes of some election. Put more formally, we get.

\begin{definition}{Domain}{domain}
	{
		Given a set of voters $N$, alternatives $A$, and conditions $C$, the domain $\mathcal{D}$ of an election is the set of all profiles $R$ such that all conditions $C$ are satisfied.
	}
\end{definition}

When we consider the domain of an election, one particular profile is the source of many impossibility results in social choice, namely, the Condorcet cycle. To understand why this profile is problematic, let us first define a notion of aggregation, the \textit{majority relation} is the preference relation we get when we compare all alternatives pairwise, and construct a preference profile from this. 

\begin{example}{Majority relations}{maj-rel}
	\begin{minipage}{0.15\linewidth}
		\begin{tabular}{ccc}
			\toprule
			$v_1$ & $v_2$ & $v_3$ \\
			\midrule
			a & b & a \\
			b & c & c \\
			c & a & b \\
			\bottomrule
		\end{tabular}
	\end{minipage}
	\hspace{2em}
	\begin{minipage}{0.70\linewidth}
		Given the profile on the left, we first start by comparing $a$ to $b$, both voters 1 and 3 prefer $a$ to $b$, thus the majority has prefers $a$ to $b$. Comparing $b$ to $c$, we see again that the majority prefers $b$ to $c$. Finally comparing $a$ to $c$ we see $a$ is again preferred. Thus the majority relation is $a \prefmaj b \prefmaj c$
	\end{minipage}
\end{example}

One property of the majority relation, that is both desirable, yet violated by the Condorcet cycle is that of transitivity. We define the transitivity on the majority relation as follows

\begin{definition}{Transitivity}{maj-trans}
	A (majority) relation is transitive, if it holds that if $a \prefmaj b$ and $b \prefmaj c$, then it must necessarily hold that $a \prefmaj c$.
\end{definition}

Intuitively it is clear why such a property is desirable, if the majority can agree on the ordering of the alternatives, it must be easier to pick a winner. Unfortunately this is not always the case, with the most famous example being the Condorcet cycle. This is a profile with 3 voters and 3 alternatives, in which all alternatives are ranked in all positions, \cref{tab: Condorcet} show a particular instance of a Condorcet cycle.

 
\begin{table}[h]
\centering
\begin{tabular}{ccc}
	\toprule
	$v_1$ & $v_2$ & $v_3$ \\
	\midrule
	a & b & c \\
	b & c & a \\
	c & a & b \\
	\bottomrule
\end{tabular}
	\caption{The Condorcet cycle, showing all alternatives in each position}
	\label{tab: Condorcet}
\end{table}

Clearly this profile present problems, as each possible outcome, would also have a majority of voters preferring another. 
 %Related literature and material (8-12 pages)
\newpage
\chapter{Theoretical Results}
\label{theory}

\textcolor{OrangeRed}{The following are propositions are not fully worked out, which is also why they are written informally, but general ideas I got from toying around with the model. I also have not spent substantial time thinking how to prove them. The point of the first two is to show that deliberation itself is not strategyproof, at least not in this form. I am not sure if whatever model of deliberation we end up creating will be (although I am highly doubtful that it will). The point is also not to say that it needs to be strategyproof, we can simply argue that deliberation not being strategyproof is preferable as voters at least have a change of recognizing some point lying etc.}

In the model of deliberation of \citet{radDeliberationSinglePeakednessCoherent2021a}, outlined in \cref{section:related_work}, aims to model deliberation, through this the aim is to show deliberation results in nicely structured profiles which allow for strategy proof voting rules. One important caveat, also given by the authors, is that all participants should honestly and truthfully participate in deliberation. We attempt to make this more formal with the following proposition.

\begin{proposition}
	Deliberation on any metric space can be manipulated if all voters can still change their mind (open-minded voters), if one voter decided not to change their mind.
\end{proposition}

\begin{proof}{(very rough sketch).} If all voters are able to change their mind, a malicious voter, call them $\hat{v}$, can simply refuse to change their mind. Since all other are still updating their preferences, if they currently hold a preference that still has a ``path" towards $\hat{v}$'s preference they will end start to agree.\end{proof}

\textcolor{Fuchsia}{The two main issues I have when I think about actually writing a proof is that the hole notion of open-minded voters is a bit vague (e.g. 0.73 is just and empiric observation), thus it would need some kind of lemma showing when voters stop being open-minded. Furthermore, I have trouble thinking about making formal statements about a deliberation step in the first place}

Unfortunately, sticking to your original preference is not the only way in which a  voter can manipulate the outcome of a deliberation. Through misrepresentation a voter could argue for a preference different from their in the hope to better the outcome. As an example imagine after 1 round of deliberation voters find themselves separated into two clusters, with mutually exclusive opinions, as in given the metric space there are no opinions between the two clusters. A strategic voter could then pretend to hold an opinion that is compatible with the other clusters and slowly move them towards their


\begin{proposition}
	Deliberation on any metric space can be manipulated if all voters can still change their mind, through misrepresentation of one current preferences.
\end{proposition}

\textcolor{Fuchsia}{I have similar issues with this one, but I think this could go two ways, either we don't let the strategic voter change their main (bias = 1), which then would build on the previous proposition. Or instead the voter can change their mind, but they will present a false preference in their announcement. I think the second approach might result in a possible complexity theory result.}

Then from a game theoretic perspective, we can model deliberation as a game, through which voters try to manipulate the outcome of the election.
 %Details of your approach (10-15 pages)
\chapter{Methods}
\label{Methods}
\lhead{\emph{Methods}} % Set the left side page header to "Symbols"

We proceed with the methods used to replicate the paper by \citet{radDeliberationSinglePeakednessCoherent2021a}, as well as the experimental setup of our own model. Links to the data used for these experiments can be found in \Cref{ethics_data}\footnote{Some references will be broken since I have simply commented out all irrelevant sections}. The programs are implemented using \texttt{OCaml}, and \texttt{Python}.


\section{Replication}
We implement the model as described in \Cref{section:related_work}, agents are only allowed strict preferences over all candidates. All experiments are done with 3 alternatives, and 51 voters. The number of voters is specifically chosen to be an odd number, as this prevents perfect ties between alternatives. We measure all evaluations relating to strict preferences, as reported by \citet{radDeliberationSinglePeakednessCoherent2021a}, in addition to those we also measure the number of Condorcet winners.

\section{Experiments} We aim to replicate the findings by the \textsc{America
in One room} experiments \cite{fishkinCanDeliberationHave2024}, to this end we
use two models. Firstly we use the adapted DeGroot model as laid out in
\Cref{sec: main model}\footnote{This section is excluded, but in summary it is
	essentially a stochastic matrix, where the entries are decided based on
	a graph and some parameters. Using this trust matrix we take power of
	this matrix, and at different powers (i.e. time steps) we left multiply
the support matrix and estimated support matrix to get the ``outcome'' of the
deliberation at that time.}, then we extend these results using our Agent Based
model. The original experiment had a control group as well as the experimental
group. To model the control group, we map all the voters onto a graph, we
explain this mapping in the next section, as well as its computational
difficulties. The experimental group is simply modelled as a densely connected
network. The trust matrix is generated based on the graph the voters are
embedded in, as such they can only trust voters they have a connection to in
the underlying graph. The distribution of this trust we control though 3
methods. Firstly, and most simply, we give all voters a bias. This bias
reflects how much of their trust they place on themselves. For example a bias
of 1 represents them placing equal trust on themselves as all other voters
combined, the actual weight on the self loop is calculated as the sum of all
incoming edges multiplied by the bias. Secondly, we have knowledge based trust,
in which a voter trusts voter $j$ more if voter $j$ is more knowledgeable. We get
the knowledge scores from the \textsc{America in One room} dataset by taking the proportion of knowledge questions they answered correctly. The
interpretation is that more knowledgeable voters would be more persuasive and
thus be more influential on other voters' opinions. Thirdly, we have
credibility based trust, where the trust a voter places on another voter is
proportional to the number of outgoing edges that second voter has. This method
becomes equivalent to placing uniform trust in all voters when all voters are
situated in a fully connected graph. If we do not use credibility or knowledge
based trust, we call this uniform trust, meaning that they treat all neighbors
the same, importantly this does not imply any specific bias value.

\subsection{Voter Mapping}
In order to simulate realistic information flow through the control group, we aim to use a natural graph structure, as well as a natural mapping from voters to nodes. Firstly, in order to generate the graph, a starting graph is taken, namely the graph of academic citations, and the TIES \cite{ahmedNetworkSamplingStatic2013} algorithm is then used to sample exactly $n$ nodes from this graph. The TIES Algorithm first samples an edge, and adds both the source and target node to the new graph, these stage is called the sampling stage. After the desired number of nodes has been reached, we proceed to the induction step, during which all the edges that exist between the sampled nodes in the original graph are added to the new graph. This algorithm allows for the use of large, natural graphs, by scaling them down to the number of nodes desired.

Once the proper graph is generated, we calculate the pairwise shortest paths between all nodes, as well as the distance in voter opinions. We then try to find a bijection between the voters and the nodes such that the difference between the shortest path and the opinion distance is minimized.

We now proceed to show that mapping voters to a graph as just described is NP-Hard, we call this problem Distance based Voter Mapping and prove two statements regarding its complexity.

\begin{theorem}
	Distance based Voter mapping is NP-Hard, when using the $\ell_2$-norm
	\label{thm:np_hard_voter_mapping_l2}
\end{theorem}

% \begin{proofc}{}
% 	The proof follows from a reduction to the Quadratic Assignment Problem.
%
% 	Assume we have matrix $A$ containing all the distances between each pair of voters, and matrix $B$ containing all shortest paths in graph $G$. Mapping the voters from $A$ to nodes in $B$ requires finding a bijection $f$ such that it minimizes the following expression:
% 	$$
% 		\begin{aligned}
% 			        & \sqrt{\sum_{i} \sum_{j} \bigl(A_{i,j} -B_{f(i),f(j)}\bigr)^2}                              \\
% 			= & \sqrt{\sum_{i} \sum_{j} A_{i,j}^2 -2A_{i,j}B_{f(i),f(j)} + B_{f(i),f(j)}^2                        }\\
% 			=       &\sqrt{ C - \frac{1}{2} \sum_{i} \sum_{j} A_{i,j}B_{f(i),f(j)} + \sum_{i} \sum_{j}  B_{f(i),f(j)}^2 }\\
% 		\end{aligned}
% 	$$
% 	 In the second to  step we note that the sum of both matrices is a
% 	 constant, thus we are minimizing in the product of entries of the two
% 	 matrices, In the final step we note that this is proportional up to a constant and positive scaling, as such the structure of the problem is identical to QAP. We
% 	 now note that the only restriction we have on the distance matrix A,
% 	 is that the voters can be embedded in a Euclidian space, however even
% 	 under this constraint QAP remains NP-Hard
% 	 \cite{queyrannePerformanceRatioPolynomial1986}.
% \end{proofc}
\begin{proofc}{}
  The proof follows from a reduction to the Quadratic Assignment Problem (QAP).

  Let $A$ be the matrix of pairwise distances between voters, and let $B$ be the matrix of shortest-path distances in the graph $G$. Mapping the voters to nodes in the graph requires finding a bijection $f$ that minimizes the following objective:
  $$
    \sqrt{\sum_{i} \sum_{j} \bigl(A_{i,j} - B_{f(i),f(j)}\bigr)^2}.
  $$
  Since the square root is a strictly increasing function, minimizing the expression above is equivalent to minimizing the sum inside:
  $$
    \sum_{i,j} (A_{i,j} - B_{f(i),f(j)})^2.
  $$
  Expanding the square gives:
  $$
    \sum_{i,j} A_{i,j}^2 - 2 A_{i,j} B_{f(i),f(j)} + B_{f(i),f(j)}^2.
  $$
  The terms $\sum A_{i,j}^2$ and $\sum B_{f(i),f(j)}^2$ are independent of $f$ (the former is fixed, the latter is a permutation of a fixed matrix), so the optimization reduces to:
  $$
    \max_f \sum_{i,j} A_{i,j} B_{f(i),f(j)},
  $$
  which is the standard form of the Quadratic Assignment Problem.

  Finally, we note that the only restriction on $A$ is that it arises from embedding voters in a Euclidean space. However, even under this constraint, QAP remains NP-hard~\cite{queyrannePerformanceRatioPolynomial1986}.
\end{proofc}
\begin{corollary}{}
	Distance Based Voter mapping is NP-Hard, when using the $\ell_1$-norm
\end{corollary}

\begin{proofc}{}
	First assume that matrices $A$ and $B$, representing the same matrices as before, are now binary matrices. Under this assumption the $\ell_1$- and $\ell_2$-norms are identical. Under this --strong-- restriction, we have reduced the problem to 0-1-QAP, which is NP-Hard \cite{nagarajanMaximumQuadraticAssignment}.
\end{proofc}

One concern with \Cref{thm:np_hard_voter_mapping_l2}, is that the data might
contain certain patterns that we might be able to exploit, though the problem statement does not
require such patterns to exist, if they do, such patterns seem
unlikely to be of much help. Take for example the case in which all voters hold
one of 2 opinions, thus we can split them into two groups of sizes $n_1, n_2$,
then the mapping algorithm effectively requires finding a partition in the graph, that
results in two sub-graphs with exactly $n_1$ and $n_2$ nodes each. This is now
the size-constrained graph partitioning problem, which is NP-Hard. Thus, given that even under
such a strong assumption the problem remains computationally intractable, we suspect that
patterns in the data are unlikely to allow for better exact solutions. 
Despite these negative results, we enlist the help of QAP-solver
\cite{virtanenSciPy10Fundamental2020}  to find solutions, using the Fast
Approximate QAP Algorithm \cite{vogelsteinFastApproximateQuadratic2015}. We
find the solver does not consistently find better solutions than random
assignment. Given the number of simulation ran, it is infeasible to attempt to
refine solutions to consistently be better than random solutions.


\subsection{DeGroot extension}

The first experiments we perform concern the DeGroot model. These experiments
consist of two parts. Firstly we search the parameter space to identify
parameters that best replicate the data, using Bayesian Parameter Estimation.
For this we use data from the \textsc{America in One room} experiment as
described in \Cref{section:related_work}. Though this data does not provide
full preference rankings over the candidates, it does provide data on voters'
opinions on 6 different topics of political discussion, such as climate change
and immigration. Using these opinions, we are able to generate potential
candidates, this is done either by selecting a voter and creating a candidate
with identical opinions, or by pooling 10 voters\footnote{This is arbitrary,
	and it might be good to look into this, but in my opinion this is low priority
	for now. It might also be useful to keep the candidates constant over the
	course of an experiment}
and creating an average of their opinions. Using these
simulated candidates we are able to create preference rankings using the
$\ell_1$-norm. To model the difference between the deliberation and control
group we change the topology of the graphs voters in the respective groups are
situated in. As mentioned before, the deliberation groups will be embedded in a fully connected
graph, while the control groups will be placed on the graph of academic
citations in physics \cite{nr}\footnote{It might also be useful to compare to
different graphs, but for now it seems okay to mention the graph's statistics
and how they compare to other social networks.}, this graph is small enough to
allow sampling of the graph for each simulation. Since the original data
provides group numbers for candidates who participated in the deliberation, we
also experiment with replicating these groups as opposed to randomly grouping
voters together.


We measure whether the final profiles are cyclic, whether they have a Condorcet
winner, home many unique profiles there are, and their proximity to being
single-peaked. Proximity to single-peakedness is measured in two ways. When the
simulation size allows for it, we measure the proximity in terms of the number
of voters that would need to be removed for the full profile to become
single-peaked. This particular notion is NP-complete
\cite{erdelyiComputationalAspectsNearly2013}, though it allows for a
2-approximation, we use the method based on an ILP solver, as implemented in
PrefTool \cite{PrefLibPreflibtools2025}. The other notion of proximity is the proximity in
terms of the number of candidates that need to be removed for the profile to
become single-peaked. This can be done in $\mathcal{O}(|V| \cdot{} |C|
^3)$\cite{przedmojskiAlgorithmsExperimentsNearly}, though the implementation we
use is that of the \texttt{PrefTools} library \cite{PrefLibPreflibtools2025}, which implements
a slower $\mathcal{O}(|V| \cdot{} |C|^5)$ algorithm
\cite{erdelyiComputationalAspectsNearly2013}. 




We aim to find values for all parameters that minimize the error of the model,
conditional on the number of voters and candidates. For this we define the
error of our model as the (normalized) difference of the proportion of cyclic
profiles, the proportion of simulation containing a Condorcet winner, and the
proximity to single-peakedness using the voter based notion if possible and the
candidate based notion. With these parameters, we argue that model captures the
learning process. We then proceed to analyze convergence behavior under these
optimal parameters, for this analysis, all configurations are run 100 times.


\renewcommand{\arraystretch}{1.2}
\begin{table}
	\centering
	\begin{tabular}{p{4cm}p{0.65\linewidth }}
		\toprule
		Parameter & Description  \\
		\midrule
	\texttt{Number of Voters} & The number of voters in the simulation, representing either the deliberation group, or the control population.\\
	\texttt{Number of Candidates}  & The number of candidates to be voted on. \\
	\texttt{Candidate Generator} & The way the candidates are generated. Either a random voter is selected for each candidate, or 10 random voters get averaged into one candidate.\\
	\texttt{Bias}& The bias all voters have towards their own opinion. \\
	\texttt{Time steps}& The number of deliberation ``steps'' the voters undergo.\\
	\texttt{Group}& Use the original groups.\\
	\texttt{Credibility}& Distribute trust based on credibility.\\
	\texttt{Knowledge}& distribute trust based on knowledge.\\
		\bottomrule
	\end{tabular}
	\caption{The parameters of the DeGroot learning based model, as well as their descriptions}
\end{table}

Given the best configurations, we will analyze the behavior of the model to
understand the convergence on opinion. To this end, we measure the change in the trust
matrix, as well as the distance between each voter's pre- and post-deliberation
preferences using the KS and CS distances. We first aim to find the number of
deliberative steps are needed for convergence, which we define as the moment where
the largest change in the trust matrix is smaller than some $\epsilon$. Then we
hope to understand how individual voters' opinions change by looking at the
final state of the trust matrix.

Finally, we use sensitivity analysis to investigate which parameters have the
strongest effect on the model, using Sobol indices to get the first and second
order effects. \footnote{This I have not had the time to implement in code yet,
as it requires a little restructuring of how the model gets its parameters.}

% \subsection{Agent Based Model}
% \begin{enumerate}
% 	\item List Graph used, neighbor selection procedure
% 	\item List parameters to be varied
% 	      \begin{itemize}
% 		      \item Hyper parameters: trust update factors, bias factors etc.
% 	      \end{itemize}
% 	\item Mention metrics of interest
% \end{enumerate}

% \subsection{Analysis}
% \begin{enumerate}
% 	\item Explain data set, as well has what a proper simulation should look like
% 	      \begin{itemize}
% 		      \item Similar trends for control vs treatment $\to$ Find pivotal voters to maximally disperse information?
% 	      \end{itemize}
%
% 	\item Statistical Tests
%
% 	      \begin{itemize}
% 		      \item Effect of single issue voters (e.g. all share similar importance vectors, for example as result of recent event) on single-peakedness
% 		      \item Effect of difference graphs, twitter vs academia etc.
% 		      \item Condorcet winners?
% 		      \item Num alternatives vs proximity to single peakedness
% 	      \end{itemize}
% 	\item sensitivity analysis
% 	      \begin{itemize}
% 		      \item Explain sensitivity analysis, Sobol indices
% 	      \end{itemize}
% \end{enumerate}

 %Details of your approach (10-15 pages)
\newpage
\chapter{Experimental Results}
\label{experiment_results}
\lhead{\emph{Experimental Results}}
We first present a full replication and extension of the work by
\citet{radDeliberationSinglePeakednessCoherent2021}. Then we present the
simulations based on our model of meta-deliberation, as well as the results of
the sensitivity analysis on both models. All code for the replication, main
experiment and visualizations can be found in
\href{https://github.com/amirsahrani/master_thesis}{this Repository}.


\section{Replication}
We are able to fully replicate the results found by
\citet{radDeliberationSinglePeakednessCoherent2021},  in \Cref{fig:rep_cyclic}
we see that while the bias is less than 0.73, all metric results in a-cyclic
preferences. We also replicate the behavior of the KS metric, where biases in
the range of 0.73-0.85, show even some initial a-cyclic profiles can become
cyclic. \Cref{fig:rep_count} Further explains this by showing that within this
range we always observe 3 unique profile for the KS metric, while DP and CS
have already settled on 6 profiles, thereby representing all possible
preferences. \Cref{fig:rep_condorcet} shows KS introduces ambiguity in the case
that there was a Condorcet winner, resulting in losing the original nice
profile. Finally, the proximity to single-peakedness shows a slightly more
positive note for the KS metric, showing that while the DP and CS bottom out to
the minimum proximity to single-peakedness, KS stays relatively close. Though
this should be taken with a grain of salt, as it is likely a consequence of the
unique preferences being smaller.

\begin{figure}[htbp]
	\centering
	\begin{minipage}{0.45\textwidth}
		\centering
		\includegraphics[width=\textwidth]{Figures/cyclic_proportion_Proportion.pdf}
		\caption{The proportion of cyclic profiles remaining, 0 indicating that no cyclic profiles were present after deliberation.}
		\label{fig:rep_cyclic}
	\end{minipage}\hfill
	\begin{minipage}{0.45\textwidth}
		\centering
		\vspace{-9pt}
		\includegraphics[width=\textwidth]{Figures/unique_Unique Preferences.pdf}
		\caption{Number of unique preferences at the final step of deliberation.}
		\label{fig:rep_count}
	\end{minipage}

	\vspace{1em}

	\begin{minipage}{0.45\textwidth}
		\centering
		\includegraphics[width=\textwidth]{Figures/condorcet_proportion_Proportion.pdf}
		\caption{The proportion of Condorcet winners left after deliberation, value above one indicate Condorcet winners emerging during deliberation}
		\label{fig:rep_condorcet}
	\end{minipage}\hfill
	\begin{minipage}{0.45\textwidth}
		\centering
		\vspace{-9pt}
		\includegraphics[width=\textwidth]{Figures/sp_proximity_PtS.pdf}
		\caption{Proximity to single-peakedness after deliberation. Proximity to single-peakedness as defined in \Cref{section:related_work}.}
		\label{fig:rep_single_peaked}
	\end{minipage}
\end{figure}

\newpage
\section{DeGroot Model}
\label{degroot_results}

We now present the results of our model based on the DeGroot learning process.
The deliberation group, which is supposed to represent a small group with
deeper talks with everyone on the group, is analysed first. The deliberation
group is modelled as a dense graph, with a few voters. Though the original data
supplied group numbers, for these experiments voters were assigned to their
groups arbitrarily. In terms of the final measures, we focus on whether the
final profiles are cyclic, whether they have a Condorcet winner, home many
unique profiles there are, and their proximity to being singly peaked.
Proximity to single peakedness is measured in two ways. When the simulation
size allows for it, we measure the proximity in terms of the number of voters
that would need to be removed for the full profile to become singly peaked.
This particular method is NP-complete
\cite{erdelyiComputationalAspectsNearly2013}, though it allows for a
2-approximation, we cannot reliably use it for larger groups, given the
sheer number of simulation necessary. The other notion of proximity, which we
will always measure, is the proximity in terms of the number of candidates that
need to be removed for the profile to become singly peaked. This can be done in
$\mathcal{O}(|V| \cdot{} |C| ^3)$\cite{przedmojskiAlgorithmsExperimentsNearly}, though the implementation used is
that of the \texttt{PrefTools} library \cite{preftool}, which implements a slower
$\mathcal{O}(|V| \cdot{} |C|^5)$ algorithm
\cite{erdelyiComputationalAspectsNearly2013}. 



\renewcommand{\arraystretch}{1.2}
\begin{table}
	\centering
	\begin{tabular}{p{4cm}p{0.65\linewidth }}
		\toprule
		Parameter & Description  \\
		\midrule
	\texttt{Number of Voters} & The number of voters in the simulation, representing either the deliberation group, or the control population.\\
	\texttt{Number of Candidates}  & The number of candidates to be voted on. \\
	\texttt{Candidate Generator} & The way the candidates are generated. Either a random voter is selected for each candidate, or 10 random voters get averaged into one candidate.\\
	\texttt{Bias}& The bias all voters have towards their own opinion. \\
	\texttt{Time steps}& The number of deliberation ``steps" the voters undergo.\\
		\bottomrule
	\end{tabular}
	\caption{The parameters of the DeGroot learning based model, as well as their descriptions}
\end{table}

Since the dataset used does not contain full preference rankings, we validate
the explanatory power of the model as follows. We aim to show that under
different numbers of voters and candidates and different ways to generate
candidates, we can find bias factors and deliberation times which minimize the
error of our model. Through showing these positive results for multiple
different (plausible) scenarios we argue that model does capture the learning
process. We then proceed to analyse the results in relation to this dataset,
interpreting the optimal bias values, as well as looking at the rate of
converges given "optimal" parameters. For this analysis, all configurations
were run 100 times. 


\subsection{Optimal parameters}

Between the deliberation group and the control group, if we look at the final
time step, we find that both perform best if the bias is set to be around 1,
though this differs based on the other parameters. This seems to indicate that
for both smaller and larger groups, a voter's opinion is in some sense equally
important as the of  \textit{all} other voters she comes in contact with. In
other words, it does not seem to matter how many people disagree with a voter,
her own opinion holds a constant relative importance.

Looking at the deliberation group, we show the best bias values in the following table:
\begin{table}[ht]
\centering
\begin{tabular}{ccccccc}
\toprule
$n_\text{candidates}$ & $n_\text{voters}$ & MSE (Sample) & MSE (Voter) & Bias (Sample) & Bias (Voter) \\
\midrule
3 & 9  & 0.00747 & 0.00733 & 1.3 & 1.3 \\
3 & 11 & 0.00951 & 0.00969 & 1.2 & 1.0 \\
3 & 13 & 0.00978 & 0.01080 & 1.2 & 1.2 \\
3 & 15 & 0.01409 & 0.01231 & 1.3 & 0.9 \\
5 & 9  & 0.03244 & 0.05191 & 0.8 & 1.1 \\
5 & 11 & 0.05640 & 0.05591 & 1.4 & 0.9 \\
5 & 13 & 0.07609 & 0.08720 & 1.1 & 0.8 \\
5 & 15 & 0.06716 & 0.07476 & 0.9 & 0.9 \\
7 & 9  & 0.07412 & 0.18686 & 1.3 & 1.2 \\
7 & 11 & 0.12538 & 0.17129 & 1.2 & 1.3 \\
\bottomrule
\end{tabular}
\caption{Minimum mean values at time step 151 for each candidate selection method, with corresponding bias.}
\label{tab:min_mean_bias_delib}
\end{table}

Here it is clear that generally the model performs best when both the number of
candidates and the number of voters are low. We also not that though the error
of the different candidate generators are comparable, they in general the
Sample methods seems to results in larger errors, meaning that the model is
less well able to capture circumstances where the alternative's opinions are
not represented in the deliberating population. Finally, we see that The
distribution of best biases skews to values around 1.3, thus indicating that
even while deliberating, people tend to hold their opinion to be \textit{more}
important than that of all other voters.

We investigate this discrepancy between the two candidates generation methods
now, to this end we look at the difference in error for all tested
configuration. 

% Now we T test on everything the Sample and Voter method

\subsection{Convergence}

From \Cref{theory}, we have seen that in the limit some matrices are
convergent, while some are not, in particular if the matrix is aperiodic, this
it is convergent. For the matrices in these simulations, we cannot guarantee 
aperiodicity. Thus, we resort to the following, instead of looking at the
matrices directly, we instead look at the distance between the estimated
support matrix, and the true support matrix, where the distance in the element
wise $\ell_2$ norm. We do the same for the support vectors and the true
opinions.

....


We find ....


\subsection{Single-peaked Preferences}

We now proceed to look at distance to single peaked profiles, look at both
voter removal and candidate removal. We show that for optimal bias, as
deliberation progresses we see an increase in the proximity to single
peakedness.

...
 %Evaluation/testing of your hypothesis (10-15 pages)
\newpage
\chapter{Discussion}
\label{Discussion}
\lhead{\emph{Discussion}}

\section{Conclusion}

The main goal of the thesis was to get a deeper understanding of deliberation
and its effect on preference profiles. To this end we consulted the literature
(\Cref{Literature}) laying out various points of view on the goal of
deliberation. From this we follow Cohen's
\cite{cohenDeliberationDemocraticLegimitimacy2002} four tenants of
deliberation; deliberation should be \textit{free, reasoned, equal}, and  it
should aim to reach \textit{consensus}. In \Cref{theory} we show that the
deliberative procedure posited by
\citet{radDeliberationSinglePeakednessCoherent2021} cannot be strategyproof
under classic notions of strategyproofness as well as novel notion of
strategyproofness we define. We use this to add one more tenant to Cohen's four,
namely \textit{honesty}.

We then set out to mechanically understand deliberation. For this, we introduced
the DeGroot learning model, and adapted it to deliberation over opinions. We showed
NP-hardness on the $\delta$-DBVM(S) problem, and concluded that using de DeGroot model
to model sparse graphs is computationally difficult, if one wants to assign voters to nodes
based on some distance metrics.

In \Cref{experiment_results} replicated the results by
\citet{radDeliberationSinglePeakednessCoherent2021}, and we use our adapted
DeGroot model to test its predictive power on opinions using the
\textsc{America in one Room} dataset \cite{fishkinCanDeliberationHave2024}. We
conclude that though in the first time step the model can do well on the
population level, the prediction on the change in opinion for individuals was
poor. We also show that this is at least partly explain by the fact that the
DeGroot model treats all policies equally. The data showed that some topics had
large shifts in opinions, while others showed less. The DeGroot model was
unable to capture this.

Using sensitivity analysis, we showed that all parameters affected the final
predictions, but interestingly some parameters had non-significant first- and second-order
effects. We argue that this is a result of these parameters not introducing new
information. As a result, they can only affect the variance of the model by
modulating the dynamics induced by the parameters with significant first-order
effects.

Finally, we looked at the preference profiles which we simulated based on the
opinions from both the data and the simulations. We show, that similar to the
population level predictions for the PBS, the profiles based on the simulated
and true opinions start looking more similar during the first steps in the
simulation. However, after this the model converge too strongly and the
profiles of the simulated opinions become too ``nice'', in the sense that they
get closer to being single-peaked and are acyclic more frequently.

These results led us to conclude that the DeGroot learning model was overly
simplistic and therefore was unable to adequately explain individual opinion
change. As a result it is a bad approximation of what happens during human
deliberation. These patterns are also in contradiction to known results in
social psychology, where small extreme groups tend to become more extreme \cite{myersPolarizingEffectGroup1975}.

\section{Discussion}

We first present some limitations of these results. We can broadly put these
into three categories.

Firstly, given the lack of a complete data source combining pre- and
post-deliberation opinions and preference rankings as well as the opinions of
these alternatives, we have had to make many assumptions on both the positions
represented by the candidates, and well as the method by which voters generate
their preference rankings. In terms of generating candidates, our
approach is simple, and only assumes that candidates represent the opinions held
by the voters. This is however clearly a less rich process than that by which
real-world candidates are selected, where these might bring in new opinions or
have traits that are desirable, such as being good leaders or well-spoken. In
terms of voters creating a ranking over alternatives, we have gone with the
assumptions that this is done strictly through distance in opinions, similar to
what a political compass test might do. In reality however, voters might be
using different and multiple heuristics to order the candidates. Indeed if
there are numerous candidates, the ranking might not even be complete.
Therefore, distance-based measures will likely diverge from heuristics, such as
pre-selecting some list of candidates deemed acceptable.


Secondly, there are some methodological assumptions we made. These mainly
relate to the generation of the trust matrices. For all Knowledge,
Self-Knowledge, and Similarity the scores were normalized to be between 0 and
1, while the Ego score was not normalized. This results in an asymmetry that
allows Ego to increase the values in the trust matrix, where the other
parameters could not. This decision was made as we found no clear ceiling
with respect to which we could normalize the Ego score. As mentioned in \Cref{experiment_results},
this might explain why Ego resulted in the lowest error on the Population level.

The same trust matrix was used for substantive and meta deliberation. Though
from a modeling perspective this is a pragmatic solution. In reality this
assumption seems too strong. This assumption forces someone to be equally
willing to change their opinion as they are to change their perception of a
candidate's opinion, where, at least intuitively, one might expect more
willingness on the latter towards people with dissimilar opinions.


Apart from these limitations in generating the trust matrices, we also note the
noise added to the estimates of candidates' opinions is normally distributed.
Though this was done to introduce voter uncertainty, over which they could then
deliberate, normally distributed noise seems unlikely, especially for voters
that hold more extreme positions. Here we might expect that the noise is
dependent on the candidates opinions, where candidates that are more similar in
opinion to the voters, will be more accurately estimated than dissimilar
candidates. For these dissimilar candidates, it might then also be true that
this noise is skewed towards the opposite extreme w.r.t. the voter's opinion.

While we opted for the DeGroot model as a more accurate representation of human
belief updating than full Bayesian updating, the DeGroot model does have some
inherent limitations. Firstly, it does not take into account why people hold
certain beliefs, nor does it constrain what kinds of beliefs a voter can hold
at the same time. To remedy this, one might consider a framework such as
abstract argumentation theory \cite{dungAcceptabilityArgumentsIts1995}, as
this is able to model the arguments with the deliberative groups. Though, this
is theoretically nice, as it allows for formal description on why opinions and
preferences are held, not just their descriptions. From a simulation
perspective, such a framework introduces major validity questions. Firstly the
framework requires a map on the relation of all arguments, for this one does
not only need qualitative data, i.e. reported arguments by participants, but
also a method of reliably and accurately transforming these qualitative reports
to argumentative graphs. Secondly, the abstract argumentation framework does
not pose an updating mechanism, thus the method through which participants
would update their believes using this framework is unclear. Secondly, it limits
voter's belief updates to linear transformations.


Finally, we address some limitations on the real-world implications of these results.
The negative results surrounding strategyproofness in \Cref{theory} might be less
of an issue in human deliberation, as the dishonest participant could be less convincing
defending their dishonest opinion than their true opinion. As a result they might have less
total influence than if they had defended their true opinion.

In terms of modeling deliberation, we have now focussed on variables that can
clearly be measured. While this might paint a good picture of the quantitative
aspects of deliberation, in practice deliberation in humans come with rich interactions
affecting their judgement and willingness to listen among other things. If we hope to
get an accurate mechanistic model of deliberation, these qualitative aspects of deliberation
need to be studied.




\section{Future work}

Based on the limitations of this study, and the literature, we present some areas for future work.

Given the weak performance of the model, a better computational model is needed
to understand deliberation and inform the design of deliberative interventions.
We propose some extensions to the model, which might better capture human
dynamics. Most importantly, it needs to be able to show non-linear affects, and
be informed by qualitative descriptions of deliberation. One main improvement of the DeGroot model specifically could be to introduce dynamic trust matrices.
When humans deliberate, the amount of trust placed on each person is likely not
fixed over time. This can be addressed dynamic trust matrices that update according
to voter's familiarity with other voters, and possibly other factors.

Another way in which the trust matrices can be further refined is through
introducing topic-dependent trust. As some topics might be more hotly debated,
for example as a result of some recent event. These voters could generally be
more informed on these topics, and less willing to talk about other topics.
This is related to the notion of \textit{Salience} as described by
\citet{listDeliberationSinglePeakednessPossibility2013}, stating that topics
with high salience benefit less from deliberation, as participants have likely
received more information on this topic.

Furthermore, any good model will need proper data, as such a study similar to
that of \citet{fishkinCanDeliberationHave2024} is needed, where voters are
asked not only for their opinion but also their preference order. This could
also be a great opportunity to gather qualitative insights into deliberation
and the social dynamics thereof. This would also allow for testing participant's
knowledge on topics directly, hopefully giving stronger indications of voter's ability
to persuade and defend on specific topics.

 %Evaluation/testing of your hypothesis (5-8 pages)
% \input{Chapters/ConclusionsandFutureWork} % (5-6 pages)

%----------------------------------------------------------------------------------------
%    THESIS CONTENT - APPENDICES
%----------------------------------------------------------------------------------------

\addtocontents{toc}{\vspace{2em}} % Add a gap in the Contents, for aesthetics

\appendix % Cue to tell LaTeX that the following 'chapters' are Appendices

% Include the appendices of the thesis as separate files from the Appendices folder
% Uncomment the lines as you write the Appendices

\newpage
\chapter{Ethics and Data Management}
\label{edm}
\lhead{\emp{Ethics and Data Management}}
A new requirement for the thesis is that there must be a short section in which you reflect on the ethical aspects of your project. This requirement is related to one of the final objectives that a graduated student of the Master of Computational Science must meet: “The graduate of the program has insight into the social significance of Computational Science and the responsibilities of experts in this field within science and in society". You don't need to devote an entire chapter to this; a short section or paragraph is sufficient.

I acknowledge that the thesis adheres to the ethical code (\url{https://student.uva.nl/en/topics/ethics-in-research}) and research data management policies (\url{https://rdm.uva.nl/en}) of UvA and IvI.

The following table lists the data used in this thesis (including source codes). I confirm that the list is complete and the listed data are sufficient to reproduce the results of the thesis. If a prohibitive non-disclosure agreement is in effect at the time of submission ``NDA'' is written under ``Availability'' and ``License'' for the concerned data items.

\begin{table}[h]
	\centering
	\begin{tabular}{lll}
		\textbf{\begin{tabular}[c]{@{}l@{}}Short description     \end{tabular}} & \textbf{\begin{tabular}[c]{@{}l@{}}Availability \end{tabular}} & \textbf{\begin{tabular}[c]{@{}l@{}}License \end{tabular}} \\ \hline
		America In One Room                                                                        &              \url{https://doi.org/10.7910/DVN/ERXBAB}                         &     CC0 1.0                                                                             \\ \hline
	\end{tabular}
\end{table}

% Appendix A


\chapter{Additional Material}
\label{AppendixA} % For referencing this appendix elsewhere, use \ref{AppendixA}


\section{Extended Proof}
\label{AppendixA:proof} % For referencing this appendix elsewhere, use \ref{AppendixA}

We present the following extension to the proof of \Cref{prop:rad_roy_delib},
specifically for the case where the CS distance is used.

\begin{proofc}
	As in the KS and DP cases, we construct profiles \( R_1 \), \( R_j \), and \( R_1' \) such that:
	\begin{itemize}
		\item \( \operatorname{Dist}_{\text{CS}}(R_1, R_j) = 2 \) for all \( j \neq 1 \),
		\item \( \operatorname{Dist}_{\text{CS}}(R_1, R_1') = 2 \),
		\item \( \operatorname{Dist}_{\text{CS}}(R_1', R_j) = 4 \).
	\end{itemize}

	Assume voter 1 has a bias of 1, and all other voters \( j \neq 1 \) have bias 0.5.

	Let the profiles be defined as follows:
	\[
		\begin{aligned}
			R_1  & = a \pref b \pref c \pref \cdots \pref m, \\
			R_j  & = b \pref a \pref c \pref \cdots \pref m, \\
			R_1' & = a \pref c \pref b \pref \cdots \pref m.
		\end{aligned}
	\]

	Observe that \( R_1 \) differs from both \( R_j \) and \( R_1' \) by a single adjacent transposition, and hence the CS distance between them is 2:
	\[
		\operatorname{Dist}_{\text{CS}}(R_1, R_j) = \operatorname{Dist}_{\text{CS}}(R_1, R_1') = 2.
	\]

	To compute the CS distance between \( R_1' \) and \( R_j \), consider the rankings of the top three candidates:
	\[
		\begin{aligned}
			\text{Positions in } R_1': & \quad a = 1,\; b = 3,\; c = 2, \\
			\text{Positions in } R_j:  & \quad a = 2,\; b = 1,\; c = 3.
		\end{aligned}
	\]
	Then:
	\[
		\operatorname{Dist}_{\text{CS}}(R_1', R_j)
		= |1 - 2| + |3 - 1| + |2 - 3| = 1 + 2 + 1 = 4.
	\]

	This satisfies the required conditions: the misreported preference \( R_1' \) increases the distance to other voters while remaining close to the voter’s true preference \( R_1 \), making strategic manipulation beneficial under this setup.
\end{proofc}



\newpage

\section{Additional Figures}
\label{AppendixB:figure}

\begin{figure}[ht]
	\begin{center}
		\includegraphics[width=.7\textwidth]{Figures/knowledge_pbs_dist.png}
	\end{center}
	\caption{The distribution of knowledge scores for different ranges of policy-based ideology scores.}\label{fig:knowledge_pbs}
\end{figure}


% % Appendix A

\chapter{Extended Proofs}

\label{AppendixA} % For referencing this appendix elsewhere, use \ref{AppendixA}

Finally, for \emph{CS}, $R_{1}$ and $R_{j}$ stay the same, while $R_{1}^{'} = c \pref a \pref b \pref \cdots \pref m$, resulting in $\operatorname{Dist}_{\text{CS}}(R_{1}^{'}, R_{j}) = |2-2| + |1-3| + |3-1| = 4$.


%\input{Appendices/AppendixC}

\addtocontents{toc}{\vspace{2em}} % Add a gap in the Contents, for aesthetics

\backmatter

%----------------------------------------------------------------------------------------
%    BIBLIOGRAPHY
%----------------------------------------------------------------------------------------

\label{References}

\lhead{\emph{References}} % Change the page header to say "Bibliography"

\bibliographystyle{plainnat} % Use the "unsrtnat" BibTeX style for formatting the Bibliography

\bibliography{references}

\end{document}
