\documentclass[10pt,dvipsnames]{beamer}
\usepackage{xcolor}
\usepackage{enumitem}
\usepackage{bookmark}
\usepackage{setspace}
\usepackage{Oswald}


\title{Modeling Meta-Agreement through Deliberation:\\ An Adaptation of the DeGroot Model}
\author[Amir Sahrani]{Amir Sahrani\\
Computational Science MSc\\
University of Amsterdam}
\date{24 July 2025}

\setbeamertemplate{frametitle}{%
    \vspace{0.5em}% Space before the title
    {\centering{\Large\textcolor{RoyalPurple}{\insertframetitle}}\par}
    \vspace{1em}% Space after the title
}

\setbeamertemplate{footline}{
  \hfill%
  \usebeamercolor[fg]{page number in head/foot}%
  \usebeamerfont{page number in head/foot}%
  \setbeamertemplate{page number in head/foot}[framenumber]%
  \usebeamertemplate*{page number in head/foot}\kern1em\vskip2pt%
}
\beamertemplatenavigationsymbolsempty

\definecolor{ICLLightGrey}{RGB}{245, 245, 245} 

\begin{document}

\setstretch{1.3} % Reset the line-spacing to 1.3 for body text (if it has changed)
\begingroup
\setbeamercolor{background canvas}{bg=ICLLightGrey} % Slide background color
\setbeamercolor{title page title}{fg=RoyalPurple} % Title text color
\setbeamercolor{title page subtitle}{fg=RoyalPurple} % Subtitle text color
\setbeamercolor{author}{fg=OrangeRed} % Author(s) text color
\setbeamercolor{date}{fg=OrangeRed} % Date text color
\frame{\titlepage} % Output the title page with no footer ('plain') and vertically distributed text ('s')
\newcommand{\slidetitle}[1]{

	{\textcolor{RoyalPurple}{\Large #1}\par}
}

\begin{frame}

	\slidetitle{Choosing a contractor}

        Alice thinks \textit{i3} delivers the best quality, and has no monetary
        restrictions.
         
        Bob does not care as much for the quality, and thinks \textit{dwm} is
        cheapest.
         
        Charlie thinks \textit{qtile} delivers the best quality, and is the cheapest
        contractor.
        
    \vspace{1em}
    
	Alice: $\text{i3} \succ \text{qtile} \succ \text{dwm}$

	Bob: $\text{dwm} \succ \text{i3} \succ \text{qtile}$

	Charlie: $\text{qtile} \succ \text{dwm} \succ \text{i3}$


\vspace{1em}


	Instead of directly trying to pick a winner, we ask them to talk a bit before.

\end{frame}

\begin{frame}

	\slidetitle{Choosing a contractor}

	They realize that Bob and Charlie hold mutually exclusive beliefs,  
	namely \textit{dwm} and \textit{qtile} cannot both be the cheapest options. It turns out  
	that \textit{qtile} is running a deal!
	\vspace{2em}

	Alice: $\text{i3} \succ \text{qtile} \succ \text{dwm}$

	Bob: $\text{qtile}  \succ \text{dwm} \succ \text{i3}$

	Charlie: $\text{qtile} \succ \text{dwm} \succ \text{i3}$

	\vspace{2em} These preferences are now single-peaked. This now allows  
	us to pick \textit{qtile} as the winner, using the rule of picking the median alternative.

\end{frame}


\begin{frame}
	\slidetitle{Motivation}

        The example introduces the two main objects of interests of this work.
        Namely, deliberation and strategyproofness.

	Following work by List (2002) proposing deliberation
	as a mechanism of enforcing shared `issue-dimensions'.

	We set out to find a formal description of deliberation, as well as a
	mechanistic computational model.

	Using this, we hope to be able to understand deliberation and the process by which it increases shared issue-dimensions

\end{frame}

% \begin{frame}

% 	\slidetitle{Outline}

% 	\begin{itemize}[label=$\circ$]
% 		\item Short history of deliberation theory
% 		\item Models of deliberation
% 		\item Negative results
% 		\item Our model
% 	\end{itemize}


% \end{frame}

\begin{frame}

	\slidetitle{Classical Result}
        
        
        \begin{theorem}{Gibbard-Satterthwaite theorem (1973, 1975).}
            There exists no resolute social choice function for elections with 3 or more candidates that is surjective, strategyproof and non-dictatorial.
        \end{theorem}

        \textit{Solution}: Single-peaked preferences allow for the median-voter rule to satisfy all axioms. (Black 1948)


\end{frame}

\begin{frame}

	\slidetitle{Deliberation, a political science perspective}

		Tenets of deliberation (Cohen, 2002): Free, Equal, Reasoned, Consensus.
        
            List (2002): Meta-agreement, unanimous consensus too strong on substantive agreement.
                Meta-agreement requires three hypotheses to be satisfied.
            

\end{frame}



\begin{frame}

	\slidetitle{Honesty}
        Can deliberation really be considered strategyproof? 

        Using definitions of Rad and Roy (2021), we show that this is in fact not the case.
        
        Trivially: people be artificially more stubborn

        Fixed bias: People can misreport preferences, and minimize outcomes under different metrics.

	New tenet of deliberation: Honesty
\end{frame}

\begin{frame}

	\slidetitle{America in one Room (2020)}

	A large-scale deliberative experiment measuring the impact of structured political discussion on voter attitudes.

        \begin{itemize}
		\item [-] Large scale deliberative experiment
		\item [-] Measured pre- and post-intervention opinions of participants
		\item [-] Deliberation caused participants to be more likely to vote, have more favorable opinions of political rivals, and be more likely to support Joe Biden
	\end{itemize}

	We use the data from this experiment to validate our own model.



\end{frame}

\begin{frame}

	\slidetitle{Our model: The Adapted DeGroot model}


        DeGroot model reduces the group dynamics of opinion change to a network of trust.

        Shown to be more accurate at modeling human belief updating than Bayesian updating

        Meta-agreement $\to$ arguing over positions of candidates.
        
        A deliberation step can be modeled as a matrix multiplication

        $$P^{(1)} = TP^{(0)}$$
\end{frame}

\begin{frame}
\slidetitle{Our model: Computational Complexity}

% We analyze the complexity of comparing two shortest-path matrices under relabeling:

\begin{problem}[$\delta$-DBVM($S$)]
Given: $A, B \in S^{n \times n}, k\in\mathbb{R}_{\geq 0}$ \\
Decision: Does there exist a bijection $f: [n] \to [n]$, such that $\delta(A, f(B)) \leq k$?
\end{problem}


\begin{theorem}
$\delta$-DBVM($S$) is NP-complete, for $\delta \in \{\ell_1, \ell_2\}$ and $S \in \{0,1\}$
\end{theorem}

\textit{Sketch:} We reduce to the 0-1 MAX-QAP
\end{frame}

\begin{frame}

	\slidetitle{Experimental Setup}

        Using data from the \textsc{America in one room} experiment to inform each voter's opinion.

        We randomly generate candidates by averaging over 1 or 10 random voters, voters inaccurately judge candidate positions.

        Sample $n$ random voters for a deliberation group, forming a dense network.

        Trust matrices generated using: 
        \begin{itemize}
            \item[-] Knowledge
            \item[-] Ego
            \item[-] Similarity
            \item[-] Bias
        \end{itemize}


\end{frame}

\begin{frame}

	\slidetitle{Results - PBS}
    
        \begin{figure}
            \centering
            \frame{\includegraphics[width=0.95\linewidth]{Figures/pbs_scores.png}}
        \end{figure}
\end{frame}

\begin{frame}

	\slidetitle{Results - Breaking down PBS}
    
        \begin{figure}
            \centering
            \frame{\includegraphics[width=0.7\linewidth]{Figures/per_topic_change.png}}
        \end{figure}
\end{frame}

\begin{frame}

	\slidetitle{Results - Errors}
    
        \begin{figure}
            \centering
            \frame{\includegraphics[width=0.9\linewidth]{Figures/errors_binned.png}}
        \end{figure}


\end{frame}

\begin{frame}

	\slidetitle{Results - Sensitivity}
    
        \begin{figure}
            \centering
            \frame{\includegraphics[width=0.9\linewidth]{Figures/senstivity_analysis.png}}
        \end{figure}


\end{frame}

\begin{frame}

	\slidetitle{Results - Proximity to Single-Peakedness}
    
        \begin{figure}
            \centering
            \frame{ \includegraphics[width=0.9\linewidth]{Figures/pst_measures.png} }
        \end{figure}


\end{frame}
\begin{frame}

	\slidetitle{Limitations}

        The model has poor individual predictive power.
        
        Trust matrices are likely not realistic.

        Voters' inaccuracy in judging candidates is likely not normally distributed.

        \vspace{3em}
        {\Large\textcolor{RoyalPurple}{Future work}}

        Richer computational model
            \begin{itemize}[label=-]
                \item Dynamic trust
                \item Non-linear interactions
                \item Negative influence
                \item Agent behavior based on social science literature
            \end{itemize}
            
        Proper data

\end{frame}


\begin{frame}

	\slidetitle{Conclusion}

        Deliberation: 
        \begin{itemize}[label=-]
            \item Increases (proximity to) single-peakedness
            \item Does not ensure strategyproofness in a broad sense 
        \end{itemize}
        

        Adapted DeGroot model: 
        \begin{itemize}[label=-]
            \item Poor predictor of individual opinion change
            \item Limited application with separate network data
            \item Requires more nuanced interactions
        \end{itemize}

\end{frame}

\endgroup

\end{document}
