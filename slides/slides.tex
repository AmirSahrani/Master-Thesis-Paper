\documentclass[10pt,dvipsnames]{beamer}
\usepackage{xcolor}
\usepackage{enumitem}
\usepackage{bookmark}
\usepackage{Oswald}

\title{Modelling Meta-Agreement through Deliberation: An Adaptation of the DeGroot Model}
\author{Amir Sahrani}
\date{13-01-2025}

\setbeamertemplate{frametitle}{%
    \vspace{0.5em}% Space before the title
    {\centering{\Large\insertframetitle}\par}
    \vspace{1em}% Space after the title
}
\definecolor{ICLLightGrey}{RGB}{245, 245, 245} 

\begin{document}
\begingroup
\setbeamercolor{background canvas}{bg=ICLLightGrey} % Slide background color
\setbeamercolor{title page title}{fg=OrangeRed} % Title text color
\setbeamercolor{title page subtitle}{fg=RoyalPurple} % Subtitle text color
\setbeamercolor{author}{fg=OrangeRed} % Author(s) text color
\setbeamercolor{date}{fg=OrangeRed} % Date text color
\frame[plain, s]{\titlepage} % Output the title page with no footer ('plain') and vertically distributed text ('s')
\endgroup
\newcommand{\slidetitle}[1]{

	{\textcolor{RoyalPurple}{\Large #1}\par}
	\hspace{3em}
}

\begin{frame}

	\slidetitle{Working Example}


	Alice: $a \succ b \succ c$, thinks $a$ is most tasty, no monetary restrictions

	Bob: $c \succ a \succ b$, does not care as much for taste, thinks $c$ is cheapest

	Charlie: $b \succ c \succ a$, thinks $b$ is most tasty, also thinks $b$ is the cheapest

	\hspace{3em}

	Instead of directly trying to pick a winner, we ask them to talk a bit before.


\end{frame}

\begin{frame}

	\slidetitle{After deliberation}

	They realize that Bob and Charlie hold mutually exclusive beliefs,
	namely $c$ and $b$ cannot both be the cheapest options. It turns out
	that $b$ is actually cheaper!
	\vspace{2em}

	Alice: $a \succ b \succ c$

	Bob: $b \succ a \succ c$

	Charlie: $b \succ c \succ a$

	\vspace{2em} These preferences are now single-peaked. This now allows
	us to pick $b$ as the winner, using the strategyproof rule of picking the median alternative.
\end{frame}

\begin{frame}
	\slidetitle{Motivation}

	Following work by list etc. we propose deliberation
	as a mechanism of enforcing ``shared realities''.


	We set out to find a formal description of deliberation, as well as a
	mechanistic computational model.

	Using this we hope to be able to understand deliberation and inform
	interventions on cultivating shared realities.

\end{frame}

\begin{frame}

	\slidetitle{Outline}

	\begin{itemize}[label=$\circ$]
		\item Short history of deliberation theory
		\item Models of deliberation
		\item Negative results
		\item Our model
	\end{itemize}


\end{frame}

\begin{frame}

	\slidetitle{Deliberation, a political science perspective}

	\begin{minipage}[t]{0.45\linewidth}
		Tenets of deliberation (Cohen, 2002):
		\begin{itemize}[label=$\circ$]
			\item Free
			\item Equal
			\item Reasoned
			\item Consensus
		\end{itemize}
	\end{minipage}
	\begin{minipage}[t]{0.5\linewidth}
		\begin{itemize}
			\item [] List (2002): Meta-agreement, unanimous consensus too strong on substantive agreement.

			\item []Meta-agreement requires three hypotheses to be satisfied


			\item []Deliberation in mini republics increases voter knowledge and judgement


		\end{itemize}
	\end{minipage}

\end{frame}

\begin{frame}

	\slidetitle{America in one Room (2020)}

	\begin{itemize}
		\item [] Large scale deliberative experiment
		\item [] Measured pre- and post-deliberation opinions of participants
		\item [] Deliberation caused participants to be more likely to vote, have more favorable opinions of alternatives, and be more likely to support Joe Biden
	\end{itemize}

	We use the data from this experiment to validate our own model.



\end{frame}

\begin{frame}

	\slidetitle{Deliberation, a modelling perspective}

	\begin{minipage}[t]{0.45\linewidth}
		Rad and Roy (2021)
		\begin{itemize}[label=$\circ$]
			\item Participants discuss preference rankings
			\item Make use of different metric spaces to calculate new opinion
			\item Participants are biased
			\item Only consensus in the form of unanimous preferences
		\end{itemize}
	\end{minipage}
	\hspace{2em}
	\begin{minipage}[t]{0.45\linewidth}
		Missing:
		\begin{itemize}[label=$\circ$]
			\item No ``meta-agreement''
			\item Updating procedure requires knowledge of large graphs
			\item Reasoned
			\item Consensus
		\end{itemize}
	\end{minipage}

\end{frame}

\begin{frame}

	\slidetitle{Notation}

	\begin{minipage}[t]{0.45\linewidth}
		Rad and Roy (2021)
		\begin{itemize}[label=$\circ$]
			\item Participants discuss preference rankings
			\item Make use of different metric spaces to calculate new opinion
			\item Participants are biased
			\item Only consensus in the form of unanimous preferences
		\end{itemize}
	\end{minipage}
	\hspace{2em}
	\begin{minipage}[t]{0.45\linewidth}
		Missing:
		\begin{itemize}[label=$\circ$]
			\item No ``meta-agreement''
			\item Updating procedure requires knowledge of large graphs
			\item Reasoned
			\item Consensus
		\end{itemize}
	\end{minipage}

\end{frame}


\begin{frame}

	\slidetitle{Strategyproofness}


	This leads to the next tenet of deliberation: Honesty
\end{frame}


\begin{frame}

	\slidetitle{Our model: The Adapted DeGroot model}


\end{frame}


\begin{frame}

	\slidetitle{Experimental Setup}


\end{frame}

\begin{frame}

	\slidetitle{Results}


\end{frame}


\begin{frame}

	\slidetitle{Limitations}


\end{frame}


\begin{frame}

	\slidetitle{Conclusion}


\end{frame}

\end{document}
